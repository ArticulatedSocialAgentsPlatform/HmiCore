


\section{XML encoding of VObjects}

%The VirtualObjectAdapter class is itself an extension of XMLStructureAdapter,
%so a lot of useful techniques and methods are available for all extensions of VirtualObjectAdapter and
%VisualObjectAdapter.
%(See also the documentation for XMLStructureAdapter in the parlevink.xml package)
%Although it would be possible to re implement basic XMLStructure methods ``from the ground up'' so to say,
%this is in general not a good idea. In such cases, you would need to take care of generic VirtualObject/VisualObjectAdapter data.
%Rather than doing this, it is better to re implement a few selected VirtualObjectAdapter methods, which will
%ensure that all such ``generic'' data is taken care of.
%The layout VirtualObjectAdapter encodings is like the following: (not all parts need be present)
%\begin{verbatim}
%<VirtualObjectAdapter (VirtualObject  XML attributes) >
%   <Attributes>
%   ...(encoded VirtualObject attributes)
%   </Attributes>
%   <Children>
%   ... (encoded VirtualObject parts)
%   </Children>
%</VirtualObjectAdapter>
%\end{verbatim}
%We remark that when the virtual object defines no attributes at all, then
%the \verb"<Attributes>..</Attributes>" section
%is omitted. Also, when there are no child parts at all, the \verb"<Children> ...</Children>"
%section will be omitted.
%Why did we say that the encoding is ``like the following''?
%Because all of it can be modified for classes that derive
%from VirtualObjectAdapter.
%For instance, the  names of the tags can be chosen differently, and the \verb"<Children>" tag could even be
%omitted altogether.
%As a starter: the XML tag (``VirtualObjectAdapter'') is not really appropriate for any derived class,
%and should be replaced by an better tag name, preferably some that mimics the class name.
% Say you have a class called ``project.Vproject.MyVObject'', then the preferred tag would be ``MyVObject''.
%This is done as usual for all XMLStructureAdapters:
%by redefining the public String getXMLTag() method, which the tag name for that class, and  which would
%be the String ``MyVObject'' in the example above.
%It is also highly desirable to register this tag name with the parlevink.xml.XML class, in order to
%enable automatic reconstruction of objects from XML code. (See the parlevink.xml documentation)
%The next step is that the derived class might need to add its own XML attributes to the
%start tag (i.e. within the \verb"<MyVObject ......>") section.
%The appropriate way to do this is to (re)implement the \verb"appendAttributeString" method.
%This method should start with calling \verb"super.appendAttributeString(..)", which will take care
%of all attributes contributed by parent classes. We give a small sample of this type of code:
%\begin{verbatim}
%public StringBuffer appendAttributeString(StringBuffer buf) {
%   super.appendAttributeString(buf);
%   appendAttribute(buf, "myAttribute1", attr1.x, attr1.y);
%   if (attr2 != null) appendAttribute(buf, "attr2", attr2);
%      appendAttribute(buf, "scale", sx, sy, sz);
%   return buf;
%}
%\end{verbatim}
%Here we have assumed that we want to add two attributes, called ``attr1'' and attr2''.
%The first one has two fields, both of which are floats, and the second attribute has a (single) String
%typed value. In general, you must append to the ``buf'' StringBuffer, but in many cases you
%can use the appendAttribute method, defined in XMLStructureAdapter. (This method allows all kinds
%of arguments, and will encode them, see the parlevink.xml documentation)
%
%The counterpart of this is of course, how to \emph{decode} attributes.
%A similar strategy like the one for encoding is available: reimplement the \verb"decodeAttribute" method, and for all attributes that
%you don't recognize, call super.decodeAttribute().
%The boolean value that is returned denotes whether the attribute name was recognized. An example:
%\begin{verbatim}
%public boolean decodeAttribute(String attrName, String valCode) {
%   if (attrName.equals("attr1")) {
%      float[] t = XMLStructureAdapter.decodeFloatArray(valCode, null);
%      setAttr1(t);
%      return true;
%   } else if (attrName.equals("attr2")) {
%      attr2 = valCode
%      return true;
%   } else {
%      return super.decodeAttribute(attrName, valCode);
%   }
%}
%\end{verbatim}
%decodeAttribute will be called for each separate attribute in turn;
%it checks for the attribute name, and if it is recognized
%the String typed attribute value must be decoded. In general that would involve converting
%the ``valCode'' String to the appropriate data type.
%Note that XMLStructureAdapter includes a few methods that take care of annoying problems like
%dealing with sequences of numbers. (For example, if the attribute would look like \verb/attr1="11.1 33.3"/,
%then decodeAttribute will be called with the Strings ``\verb/attr1/.'' and ``\verb/11.1 33.3/''.
%The decodeFloatArray would turn the latter String into a Float array \verb"t" with two elements, and the
%\verb"setAttr1(t)" in the example above is supposed to set the value of the attribute, using such an array value.
%
%The next modification you would like to make to the generic VirtualObjectAdapter layout is
%to add what is called ``contents'', or ``PCDATA'' in XML terminology.
%This is the section between the \verb"<MyVObject>" tag and the \verb"</MyVObject>" tag.
%The default is to include the encoding of the VirtualObject attributes (not to be confused with the
%XML style attributes \emph{within} the \verb"<MyVObject...>" tag), followed
%by the encoding of child parts.
%The code fragments responsible for this are the following:
%\begin{verbatim}
%public StringBuffer appendContent(StringBuffer buf, int tab) {
%   appendVOContent(buf, tab+TAB);
%   appendChildren(buf, tab+TAB);
%   return buf;
%}
%
%public StringBuffer appendVOContent(StringBuffer buf, int tab) {
%   if (attributes != null) {
%      buf.append('\n');
%      attributes.appendXML(buf, tab);
%   }
%  return buf;
%}
%\end{verbatim}
%From this the reimplementation strategy follows:
%\begin{itemize}
%\item If you want to add some ``content'', in the form of XML elements, \emph{after} the VirtualObjectAdapter
%Attribute section, but \emph{before} the children section, then re implement just the \verb"appendVOContent" method, like for example:
%\begin{verbatim}
%public StringBuffer appendVOContent(StringBuffer buf, int tab) {
%   super.appendVOContent(buf, tab)
%   // add MyVObject content
%   return buf;
%}
%\end{verbatim}
%\item If you want to reorder the content more drastically, you will have to re implement the
%\verb"appendContent" method.
%\end{itemize}
%
%Finally, we must discuss how this extra content is to be \emph{decoded}.
%Again, you should re implement the appropriate VirtualObjectAdapter method,
%which in this case is the \verb"decodeXMLElement" method.
%To understand the role of this method within the decoding process, we provide the
%\verb"decodeContent" and the \verb"decodeXMLElement" methods from VirtualObjectAdapter
%itself:
%
%\begin{verbatim}
%public void decodeContent(XMLTokenizer tokenizer) throws IOException {
%   while (tokenizer.atSTag()) {
%      boolean decoded = decodeXMLElement(tokenizer, tokenizer.getTagName());
%   }
%}
%
%public boolean decodeXMLElement(XMLTokenizer tokenizer, String stag)
%throws IOException {
%   if (stag.equals(attributesTag)) {
%      allocate();
%      attributes.readXML(tokenizer);
%      return true;
%   } else if (stag.equals(childrenTag)) {
%      tokenizer.takeSTag();
%      while (tokenizer.atSTag()) {
%         addChild((VirtualObjectAdapter) XML.createXMLStructure(tokenizer));
%      }
%      tokenizer.takeETag(childrenTag);
%      return true;
%   } else {
%      Console.println("Skipping unknown VirtualObject tag: " + stag);
%      tokenizer.skipTag();
%      return false;
%   }
%}
%\end{verbatim}
%
%These code fragments suggest how to re implement the \verb"decodeXMLElement" method:
%The code should be like the following:
%\begin{verbatim}
%public boolean decodeXMLElement(XMLTokenizer tokenizer, String stag)
%throws IOException {
%   if (stag.equals(MyContentTag1)) {
%      // process content element type 1
%      return true;
%   } else if (stag.equals(MyContentTag2)) {
%      // process content element type 2
%      return true;
%   } else {
%      return super.decodeXMLElement(tokenizer, stag);
%   }
%}
%\end{verbatim}
%
%The VirtualObject base class assumes that the \verb"<Children>" tag
%is used to include a number of child VirtualObject elements.
%This tag can be renamed, or it can be omitted.
%This can be done by re implementing the \verb"getChildrenTag()" method.
%The default VirtualObject implementation returns the String \verb"Children".
%If the method returns null, the encoding will omit the Children tag.
%Note that extensions of VirtualObject or VisualObject can introduce any sort of elements
%outside any ``Children'' section, including elements that should be regarded as child elements.
%It is important that such elements are included as ``child'' by calling the
%addChild(VirtualObject) method from the VirtualObjectAdapter superclass.
%\begin{verbatim}
%   public String getChildrenTag() {
%      return CHILDRENTAG;
%   }
%\end{verbatim}
%
%
%\section{Visualization}
%
%Although VisualObjects appear to be ``visual'' objects, this is only partly true.
%It is more appropriate to think of VisualObjects as VirtualObjects that have one (and possibly more)
%visualization attached to it. The dominant visualization technique at this point in time is based on
%Java3D/X3D techniques. An alternative, as yet not fully implemented, is a Java2D visualization technique.
%VisualObjects do have a number of attributes and properties that are shared among
%all visualizations: they possess a position and an orientation in (3D) space.
%Because of the hierarchical structure of VisualObjects, the position and orientation are really
%\emph{relative} to the position and orientation of some parent VisualObject, and for
%this reason these attributes are called \emph{translation}, rather than position, and
%\emph{rotation}, rather than  orientation. (This terminology stems directly from Java3D, VRML, and X3D)
%The coupling between a VisualObject and its visualization(s) has two aspects: the programming
%API, and the XML/X3D encoding aspect.
%To start with the latter: this is rather easy. The XML encoding of VisualObjects allows
%XML content to be (children) VisualObjects, as well as Visualizations.
%The latter are regarded as the Visualization that belongs to that VisualObject.
%For instance, consider the following example:
%\begin{verbatim}
%   <VisualObject id = "box1"
%     translation = "-1.2 0.5 0.0" rotation = "1 0 0 0.4">
%         <Shape>
%            <Appearance DEF = "app1">
%                  <Material DEF="m1" diffuseColor="1.0 0.0 0.1">
%                  </Material>
%            </Appearance>
%            <Box size="1.0 0.3 1.0" DEF="box1">
%            </Box>
%         </Shape>
%   </VisualObject>
%\end{verbatim}
%This defines a VisualObject ``box1'', without any VisualObject parts. It does have a
%3D visualization though, in the form of an X3D Shape.
%A more complicated example:
%\begin{verbatim}
%<VisualObject id = "hammer" translation = "0.0 -0.5 0.0">
%   <VisualObject id = "cyl" translation = "1.2 0.5 0.0"
%                    rotation = "1 0 0 0.7" dynamic="true">
%      <VisualObject id = "box2"
%         translation = "0.0 1.0 0.0" dynamic = "true">
%            <Shape>
%               <Appearance >
%                  <Material diffuseColor="0.0 1.0 1.0" />
%               </Appearance>
%               <Box size = "0.4 0.4 0.6"/>
%            </Shape>
%      </VisualObject>
%      <Shape>
%        <Appearance >
%           <Material USE="m1"/>
%           <ImageTexture url="x3d/gelemuur.jpg" />
%        </Appearance>
%        <Cylinder height="1.5" radius = "0.2"/>
%      </Shape>
%  </VisualObject>
%</VisualObject>
%\end{verbatim}
%This defines a VisualObject ``hammer'' with a part called ``cylinder'', which has an X3D Shape
%visualization. (The cylinder shape, with the ``gelemuur.jpg'' texture.
%The cylinder VisualObject has itself a child VisualObject, called ``box2''.
%The latter also has a visualization, in the form of an X3D box Shape.
%Both the cylinder and the box2 object are ``dynamic'', which indicates that they can be animated.
%In this case, the cylinder could rotate, and the box2 object could rotate on top of the cylinder.
%This pattern is also found in X3D/HAnim bone structures.
%
%The programming interface is slightly more complicated.
%Because visualizations can be either 3D or 2D, and because Java3D is not hard wired into
%the environment package, including the VisualObject class, the coupling between a VisualObject
%and its visualization(s) is established via two interfaces, defined within the environment package:
%VisualGroup and Visualization.
%A complication here is that the granularity, or level of detail at the VisualObject level
%and the visualization level need not be the same. A typical situation could
%be a ``shallow'' hierarchy at the VisualObject level, worked out only as far as is necessary
%for reasoning within object ontologies, combined with a highly detailed
%X3D visualization. The reverse is also possible: it is allowed to introduce VisualObjects
%that have no visualization at all levels of detail for both 2D and 3D visualizations.
%Think for instance of a ``book'' with many pages of text that are modeled at the VisualObject level
%as well as the 2D visualization level, but where only the book as a whole is modeled at the 3D visualization
%level.
%
%A VisualGroup is the counterpart of the VisualObject that is capable of building hierarchical ``scene graphs''.
%That is, a VisualGroup has methods addVisualChild and removeVisualChild, for adding and deleting children,
%of type VisualGroup. In addition, there are methods for setting the configuration
%( i.e. the translation, rotation, and scale) of an VisualGroup, and a few other attributes.
%It also defines a getType() method, which is used to classify the type of visualization.
%A Visualization is somewhat simpler: is has a getType() method like VisualGroup, and
%it has a method getVisualGroup() that will encapsulate the Visualization within a VisualGroup object.
%A Visualization is primarily used for the visualization of some VisualObject without parts.
%A second use is to visualize some VisualObject that does have parts at the VisualObject level,
%but where there is no need for any visualization of those parts, or at least there is no need
%for any explicit coupling between the VisualObject hierarchy and the visualization
%hierarchy. A typical case could be the ``book'' example introduced above.
