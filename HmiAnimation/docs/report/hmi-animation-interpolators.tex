


\section{Interpolators}
%VisualObjects are object with a defined \emph{translation},i.e its 3D ``position'', and a \emph{rotation}
%, i.e. its ``3D orientation''. (Note we use ``translation'' and ``rotation'' as this is the
%common terminology used by  VRML, X3D, and Java3D.)
%Sometimes the 3D information might be visualized in 2D form, or might not be visualized at all,
%but conceptually everything takes place in 3D space. Therefore a translation is always a 3D vector,
%whereas a rotation is determined by a 4-tuple, either in the form of a quaternion, or else in the form
%of an axis of rotation and a rotation angle, a so called ``AxisAngle''.
%Optionally, one can define scale factors in addition to translation and rotation, but at this moment
%this is not fully integrated.
%VisualObjects can be combined in the form of hierarchical structures, via
%the parent-child relationship, inherited from the VirtualObject interface.
%A given node in this hierarchy can be seen as a root node, that has ``offspring'', in the form
%of children, children of children etc.
%The translations and rotations from a root node to one of its offspring nodes can be combined
%into resulting translation and rotation, the so called ``pathTranslation'' and ``pathRotation''.
%(Note: this works also when \emph{uniform} scaling is used, but it works no longer
%for non-uniform scaling)
%It is possible to set the translations and rotations along the
%VisualObject nodes along a path, and then to obtain the resulting pathTranslation and pathRotation.
%It is also possible to set the translation and rotation of a child node \emph{indirectly}, by
%specifying the desired pathTranslation and pathRotation of that child node.
%In general, translations and rotations (and sometimes also scaling) are specified at the same time.
%In such cases, it is sometimes easier to work with so called Configurations.
%There is a class \verb"Config3D" which is in essence a wrapper for a translation and a rotation.
%The VisualObjectAdapter class has setConfiguration methods that accept either
%a Config3D Object, or a translation and a rotation (and optionally a uniform scaling).
%
%\emph{Interpolation} is going one step further than just setting translations and rotations.
%In essence an \emph{interpolator} is a mapping from time stamps to interpolated values,
%that can be used as the basis for an animation, for instance.
%VRML/X3D/Java3D have a rather limited idea of interpolators: each interpolator controls
%one transform(group). This is also possible with VisualObjects, which can be seen as equivalent
%to the X3D/Java3D transform(group)s, as far as animation possibilities are concerned
%However, many VisualObjects have a more complicated, hierarchical structure.
%An example is the bone structure of an avatar, which can be seen as a structured VisualObject, where
%each joint in the bone structure corresponds to one (recursive) child element of the VisualObjet
%that describes the bone structure as a whole.
%In this case, the translation and rotation of the top level VisualObject describe the
%position and orientation of the avatar as a whole, within the virtual environment, whereas
%the rotations of the joints would describe the orientation of the body segments.
%In such cases, one needs an interpolator that controls many VisualObject(parts) at the same time.
%The basic (minimal) interface for interpolators is \verb"VisualObjectInterpolator".
%A VisualObjectInterpolator must have a defined start time, a defined ``target'' VisualObject, and
%methods for time interpolation.



