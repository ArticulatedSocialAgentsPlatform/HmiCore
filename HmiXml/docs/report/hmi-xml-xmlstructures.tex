\section{XMLStructures}

\verb#XMLStructure# is an Interface, defining methods for storing Objects into XML format, and retrieving them from
such XML formats. The main methods required by this interface are:

\begin{itemize}
\item \verb#toXMLString#: produces an XML representation in String format
\item \verb#getXMLTag#: returns the XML ``tag'' used to represent an XMLStructure. (The ``myxml'' String in \verb#<myxml>...</myxml>#)
\item \verb#writeXML#: writes the \verb#toXMLString# String representation to some Java printWriter.
\item \verb#appendXML#: Like \verb#writeXML#, but appends the output to some Java StringBuilder
\item \verb#readXML#: The complement of the methods above, reconstructing the Object from the XML representation
\end{itemize}

There are a number of variations of each method, dealing with formatting, or the IO used. In practice it is easiest to implement
\verb#XMLStructure# by extending the \verb#XMLStrucrureAdapter# class, so you have to implement only a few of these methods.

The \verb#XMLForrmatting# class is used for to \verb#XMLStructures#, for formatting the XML text for \verb#appendXML#methods like
\verb#toXMLString# and \verb#appendXML#. The most basic usage is where an \verb#XMLForrmatting# object keeps track of nested indentation levels
for the xml text. For more complicated examples, the \verb#XMLForrmatting# object keeps track of nested namespace regions.
 For example, consider an XML fragment like the following:
\begin{verbatim}
<test xmlns ="http://hmi.ns.test">
   <preamble>
      data
      <nested></nested>
   </preamble>
   <ns:skippedtag attr="val" xmlns:ns="http://ns">
      <innertag> chardata
         <nestedtag> chardata
             <ns:skippedtag>
                 more data
             </ns:skippedtag>
         </nestedtag>
      </innertag>
      <sometag xmlns:some="some-namespace"/>
      <ns:extratag > moredata </ns:extratag>
   </ns:skippedtag>
   <moretags>
      data
   </moretags>
</test>
\end{verbatim}
It is clear that here formatting uses indentation to show the nesting structure.
The treatment of namespaces is more complicated. For instance, there is a namespace declaration like \verb#xmlns:ns="http://ns"#.
Such declarations introduce namespace prefixes, in this case the String \verb"ns", that will be used as prefix for tags
whenever some bested XML structure belongs to the declared namespace. The \verb#XMLForrmatting# object keeps track of such declarations,
and uses a stack in order to deal with nested declarations. 

\section{XMLStructureAdapter}

The implementation of classes that implement the \verb#XMLStructure# interface has many commonalities. The \verb#XMLStructureAdapter# class
is a useful base class that can be used to implement such \verb#XMLStructure# implementations, and that takes care of the common methods.
There are two aspects: the simple baseline is that \verb#XMLStructureAdapter# implements all methods required by \verb#XMLStructure#,
and simply requires you to overwrite or implement a few methods, as is usual with ``adapter'' classes. The more sophisticated aspect is
that \verb#XMLStructureAdapter# offers a large number of ``utility'' methods that ease the implementation of XML parsers. 
Think of things like parsing a String that encodes a number of float numbers, that must be converted into a Java array or List of floats.
Or vice versa, if you have such an array or List, how to convert it to a String, with some elementary formatting applied.

\noindent
The \emph{first}, most basic though not the most versatile, way of using \verb#XMLStructureAdapter# is to extend it and to re-implement the following methods:
%
\begin{itemize}
\item \verb#readXML(XMLTokenizer)#
\item \verb#appendXML(StringBuilder buf, XMLFormatting fmt)#
\item \verb#getXMLTag()#
\end{itemize}
%
All other \verb#XMLStructure# methods are then defined in terms of these three methods, in the obvious way.
This still leaves you with the sometimes complicated task how to parse XML structures in the \verb#readXML# method,
or how to produce it in the \verb#appendXML# method.  

\noindent
The \emph{second} approach, usually the preferred one, is to re-implement
the following set of \verb#XMLStructureAdapter# methods:
%
\begin{itemize}
\item \verb#public String getXMLTag()#
\item \verb#public StringBuilder appendAttributeString(StringBuilder buf)#
\item \verb#public boolean decodeAttribute(String attrName, String valCode)#
\item \verb#public StringBuilder appendContent(StringBuilder buf, XMLFormatting fmt)#
\item \verb#hasContent()#
\item \verb#public void decodeContent(XMLTokenizer tok)#
\end{itemize}
%





