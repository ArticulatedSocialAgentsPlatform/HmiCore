
\section{Compilation}

Compilation (currently) is relevant for the following stages:
\begin{enumerate}
\item From the hmi.graphics.collada format to hmi.graphics.scenegraph format.
This step is carried out by classes inside the hmi.graphics.collada.scenegraph package.
The idea is that after the compilation, we end up with a ``neutral'', graphics format independant description of a graphics scene.
This stage is independant from the opengl related classes.
\item From the hmi.graphics.scenegraph format to hmi.graphics.opengl format.
This stage is carried out by the hmi.graphics.opengl.scenegraph package. Since it compiles
from scenegraph format, it is independent from graphics file formats.
\end{enumerate}

The first step deals with compiling a subset of the very general Collada format into
the more regular scenegraph format. The second step compiles into a low-level OpenGL format.
As can be seen, the two steps follow logically one after another, yet, the intermediate scenegraph format
is \emph{not} dependent on either the Collada format or the OpenGL related packages.
The (future) idea is to extend the approach, in particular to allow more graphic formats,
all to be compiled into the scenegraph format. In principle, the second step could also be replaced by compiling
into something else, like Java3D, OpenGL ES etcetera.

In this section we discuss both compilation steps \emph{together}, without all of the details,
in order to see how a Collada scene ends up eventually as data structures for the Jogl renderer.
















\subsection{hmi.graphics.opengl ``controllers'' and hmi.animation ``skeletons''}

The counterpart of Collada's skin controllers at the OpenGL/Jogl level is the combination
of GLSkinnedMesh objects combined with skeleton structures in the form of trees of VJoint objects.
It is important here that skeletons and VJoints are part of the hmi.animation package, \emph{not} part of any of
the hmi.graphics packages. The rationale here was to keep ``animation'' as such independent from OpenGL implementation
issues. (The hmi.animation package can even be compiled without hmi.graphics available)

The most important points:
\begin{itemize}
\item A \verb"GLBasicmesh" defines vertex attributes like vertex coordinates, texture coordinates, normals, colors,
or more general GLSL shader attributes. It does \emph{not} deal with bone structures, skeletons, skin weights etcetera.
\item A \verb"GLSkinnedMesh" is an extension of a \verb"GLBasicMesh", and adds skinning information.
\item More in particular, a \verb"GLSkinnedMesh" defines: joint indices and joint weights per vertex, vcounts,
for the number of joint/weights per vertex, joint matrices, inverse bind matrices, joint names and skeleton id.
\item The joint matrices for a \verb"GLSkinnedMesh" are \emph{references} to matrices kept elsewhere.
Typically, each joint matrix will be a reference to a (global) matrix exported by a \verb"VJoint" object.
The rules of the game are that a \verb"GLSkinnedMesh" will read but not modify its joint matrices.
Moreover, it is expected that updates to these matrices will be performed in a way that avoids concurrency problems.
In practice this means that the thread \emph{setting} the values of VJoint matrices must
synchronize by means ofa common lock or common monitor with the thread being used for mesh deformation and rendering.
\item Inverse bind matrices are part of a \verb"GLSkinnedMesh". They are multiplied with the joint matrices
by  calling the \verb"calculateMatrices" method of a \verb"GLSkinnedMesh" object. This should be done
before the \verb"deform" method is being called on such a \verb"GLSkinnedMesh" object.
Note that inverse bind matrices are always needed, even when the bind pose involves only ``trivial'' rotations.
In that case all 3X3 rotation matrices would be identity matrices, but the \emph{translation} from one joint to the next
is still part of the 4X4 (inverse) bind matrix.

\subsection {hmi.graphics.scenegraph ``controllers''}

A (hmi.grapics.)scenegraph defines \verb"GMesh" as the counterpart of Collada's skinned meshes, and opengl's deformable meshes.
A \verb"GMesh" defines normal vertex attributes like vertex coordinates etcetera, but also a special purpose
\verb"VertexWeights" attribute, joint names, s skeleton id, and inverse bind matrices.
The skeleton id is the id of some \verb"GNode" within the scenegraph, that still has to be resolved.
Once this skeleton ``root'' node has been determined, it is assumed that joint \verb"GNode"s can be located
as (recursive) children, by means of their \verb"sid" attribute, that should correspond to the joint names
from the \verb"GMesh".


\subsection{Compiling ``Controllers''}

A Collada \verb"instance_controller"  is compiled into a \verb"GMesh".
All controller information, like joint names, skeleton id, inverse bind matrices, skin mesh, and
skinning information like weights are put into the \verb"Gmesh" object. (We assume here only one skeleton per
\verb"instance_controller")
At the same time, the scene graph's \verb"GNode"s can be compiled into a similar \verb"VJoint" structure.
It has not yet been decided what to do with potential ``sharing'' within scenegraphs, that is, for the moment
we assume tree-like structures, no ``dag'' like scene graphs.
Checking and resolving the skeleton \verb"id" or the joint name \verb"sid"'s can then be done.
In the next compilation step, a \verb"GMesh" can be compiled into a \verb"GLSkinnedMesh"
\end{itemize}


\subsection{Detailed steps for compiling skinned meshes}

A Collada skinned mesh is created by means of an \verb"instance_controller", specifying an url to the geometric mesh
as well as specifying a skeleton node. The mesh geometry itself is a child of the \verb"instance_controller" node,
which is quite independent from the skeleton node. In studio-max exports the skeleton typically resides in a completely separate scenegraph branch, and is the (only) child of some ``Bip'' root node.
Example of an instance controller, referring to the \verb"CWom0023-Pelvis-node" skeleton, and
using a geometric mesh, referenced here as \verb"CWom0023-Mesh".
\begin{verbatim}
<node id="CWom0023-Mesh-node" name="CWom0023-Mesh" type="NODE">
  <instance_controller url="#CWom0023-Mesh-mesh-skin">
    <skeleton>#CWom0023-Pelvis-node</skeleton-->
      ... (instance_material)
  </instance_controller>
</node>
\end{verbatim}

The skeleton, starting at the \verb"CWom0023-Pelvis-node" node, resides in the second secengraph branch,
with root node \verb"CWom0023-Bip-node". 
\begin{verbatim}
  
<node id="CWom0023-Bip-node" name="CWom0023-Bip" type="JOINT">
  <matrix>
     0 1 0 -0.000068 
    -1 0 0  0.016062 
     0 0 1  0.905944 
     0 0 0  1
  </matrix>
  <node id="CWom0023-Pelvis-node" name="CWom0023-Pelvis" sid="Bone1" type="JOINT">
    <matrix>
      0 1 0 -0.007917 
      0 0 1  0 
      1 0 0  0 
      0 0 0  1
   </matrix>
    <node id="CWom0023-Spine-node" name="CWom0023-Spine" sid="Bone2" type="JOINT">
      ... (remainder of skeleton graph)
\end{verbatim}

Note that the skeleton is rotated (and translated as a whole), by means of the pelvis node matrix. 

The mesh itself has (in this example) no inherent transformation, but the Collada idea is that the 
\verb"CWom0023-Mesh-node" could be moved around, affecting the position and orientation of the mesh as a whole. 
So the mesh deformation process, using the skeleton joint nodes, is decoupled from the mesh transformation
as a whole. Still the translation and rotation of the skeleton root or the ``Bip'' node will also
affect this ``global'' position and orientation. 
The skeleton structure specifies a hierarchical tree of joint nodes, each with a local translation, local rotation, and possibly, local scaling. In addition the mesh defines ``inverse bind matrices'', 
and even a \verb"bind_shape_matrix" element. The latter is rather simple: it specifies
a single matrix, to be applied to the mesh geometry in order to get it into 
the so-called ``bind pose''. 
The inverse bind matrices (one per joint) are more complex. 
In essence, they specify the matrices that where needed to get the skeleton structure
into the bind pose as well. For example, consider a skeleton that would have its arms 
hanging downwards, but, on the other hand, a mesh that in the bind pose has the avatar
in the ``T-shape'' pose, with the arms stretched horizontally. 
Then the bind pose for the skeleton would rotate the arms such that they would also be in
a T-shape pose, matching the mesh bind pose. Note that the bind matrix for some joint specifies
the \emph{global} transformation for that joint in order to be in the bind pose, not just
its \emph{local} translation and rotation. 

\subsubsection{ translation steps (``under revision'') }

\begin{enumerate}
\item The Collada package includes classa that mimic the Collada structures quite closely.
The relevant classes here (from \verb"hmi.graphics.collada") are: 
\begin{itemize}
\item \verb"Instance_Controller" : defines the mesh url, and the skeleton ids, via the associated \verb"Skeleton"
\item \verb"Skeleton" : just the url of some skeleton id
\item \verb"Controller" : just a link to the \verb"Skin"
\item \verb"Skin" : the \verb"Joints", \verb"Vertex_Weights", the \verb"Bind_Shape_Matrix"
\item \verb"Joints" : defines the SID's of the joints, as well as the associated inverse bind matrices.
\end{itemize}

\item The hmi.graphics.scenegraph package represents skinned meshes using the following classes:
\begin{itemize}
\item \verb"GNode" : the basic scenegraph node, used used here to represent skeleton joint nodes.
\item \verb"GSkinnedMesh" : keeps track of skeleton ids, joint names, joint sid's, inverse bind matrices,
vertex weights, and links to the skeleton GNode nodes that represent skeleton nodes. 
A \verb"GSkinnedMesh" is an extension of a \verb"GMesh"
\item \verb"GMesh" : represents the mesh geometry, by means of vertex attributes, and vertex index data.
\item \verb"VertexAttribute" : a generic vertex attribute, like position, normal, texture coordinate, color
\item \verb"VertexWeights" : somewhat like a complicated \verb"VertexAttribute", keeping track
of vertex weights for a \verb"GSkinnedMesh". 
\end{itemize}




\item The scenegraph structures are not directly fit for rendering in an OpenGL context.
The \verb"hmi.graphics.opengl" package contains OpenGL based counterparts for scenegraph structures.
The relevant classes for skinned meshes are:
\begin{itemize}
\item \verb"GLSkinnedMesh" : defines OpenGL vertex attributes and index data, joint matrices,
inverse bind matrices. (For debugging purposes it also keeps track of joint names etc.)
\verb"GLSkinnedMesh" is an extension of \verb"GLBasicmesh" 
It knows how to \emph{deform} the mesh data from the underlying \verb"GLBasicMesh", based upon the
current joint matrices, inverse bind matrices, and vertex weights.
\item \verb"GLBasicMesh" : keeps track of vertex attributes and vertex index data.
In particular it knows how to render the mesh, using this data.
\item \verb"GLVertexAttribute" : auxiliary class, used by \verb"GLBasicMesh" to deal
with vertex attribute data.
\end{itemize}


\item The translation from Collada structures to scenegraph structures:\\
The translation methods are located within the \verb"hmi.graphics.collada.scenegraph" subpackage.
The translation here boils down down to collecting information that is spread around within the Collada
structures, and combining it into  \verb"GSkinnedMesh" objects, together
with \verb"GNode" based skeleton graphs. The transformation part of a Collada node is transformed into
a \verb"VJoint" that is kept inside a scenegraph  \verb"GNode". 
Since both \verb"VJoint" structures and \verb"GNode" structures
are hierarchical trees, we have some (unfortunate) hierarchy duplication here. 
A slight advantage is that the \verb"VJoint" based structure is based solely upon
the \verb"hmi.animation" package, and is quite independant from the \verb"hmi.graphics" packages.

Note that a single Collada mesh can consist of more than one ``primitive'', where a primitive in this context 
is a part of the mesh geometry together with a material. Such primitives are currently translate
into separate \verb"GSkinnedMesh" objects, that typically share a single skeleton structure.


The translation at the highest level compiles a complete Collada document into a \verb"GScene" object.
The latter collects global information like the root nodes of the scenegraph, but also keeps track of 
all \verb"GSkinnedMesh" es that occur within the scene.
In a second translation step these \verb"GSkinnedMesh"es are worked on in order to
get them prepare them for rendering in the OpenGL framwork, and in order to simplify animation.
For instance, exported skinned meshes are often in the wrong orientation (lying on the ground),
are not based upon the default HAnim skeleton structure and poses, might include scaling operations
which complicate matters etcetera. The \verb"GScene" methods are meant to deal with all this. 

Here is what we would like:
\begin{enumerate}
\item The avatar is standing at the origin, upright, where ``up'' means in the positive Y-direction
\item All joints have been resolved in the sense that they are coupled to the appropriate \verb"GNode".
\item the joint names are the standardized HAnim joint names, and the joint structure conforms to the
HAnim standard. 
\item The default pose, that is, the pose where all local joint rotations have been set to 
the identity matrix, is the default HAnim pose, where for instance arms are pointing downwards,
and are aligned with the Y-axis. 
\item There is no scaling. In particular, we don't like non-uniform scaling, but other forms of scaling are also
not convenient.

\end{enumerate}

In order to achieve this, the following steps are taken:

\begin{enumerate}
\item joints are resolved: skeleton ids are used to find corresponding skeleton root nodes inside
the scenegraph \verb"GNode" branches, then joint \verb"GNode"s are searched for by sid 
within the sub tree(s) defined by these root node(s).
When found the GSkinned mesh is linked to the global matrix of the VJoint of the skeleton GNode.
This direct coupling means that during rendering the GNode structure is no longer needed: the 
mesh will use the global matrix of the joint directly. 
For convenience, we copy the joint names from the skeleton nodes to the GSkinned mesh, mainly for debugging purposes.
\item In the second step, joint sids (and joint names) are \emph{renamed}, based upon a user specified
renaming list. The renaming is carried out both on the GSkinnedMeshes as well as the GNodes from the skeletons.
This ``early renaming' guarantees that later stages can assume, for instance, that HAnim joint sids/names
are being used. The original joint ids are \emph{not} renamed. 

\item the next step is to rotate and scale the mesh as a whole. Say, your mesh is specified with centimeters,
rather than with meters. Then you want to scale by a factor $1/100$. If the avatar
is lying on its nose, in the direction of the Z-axis, you want to rotate it such that the Y-axis is the ``up-axis''.
This step actually modifies the geometric attributes of the mesh
\end{enumerate}

\item The translation from scenegraph structures to opengl structures:\\

Here we translate  \verb"GSkinnedMesh" objects into  \verb"GLSkinnedMesh" objects,
\verb"GNode" objects into a \verb"GShape" objects.
The \verb"VJoint" object that represents
the transformation aspect of a \verb"GNode" are kept, and form the transformation scenegraph during rendering.



\end{enumerate}

