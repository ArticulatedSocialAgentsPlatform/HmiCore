
\section{OpenGL}

The hmi.graphics.opengl package defines a basic render engine based upon the Jogl binding for OpenGL.

\begin{itemize}
\item GLRenderContext: wrapper class for a Jogl GL object. This is the Jogl defined interface to OpenGL functionality.
In addition a GLRenderContext can define extra render state information, such as
the current \emph{renderpass}.
\item GLRenderObject: Interface for all objects that can be ``rendered'' using an OpenGL render context.
The two methods: glInit and glRender are called with a GLRenderContext parameter that includes
a Jogl GL object with what is called a \emph{current} OpenGL context.

\item GLRenderList: Basically a List of GLRenderObjects, that can be treated as a GLRenderObject itself.

\item GLShape: An GLRenderObject that defines, by means of a user specified GLRenderList,  an OpenGL render state,
that defines geometry by means of a second user defined GLRenderList, and
that defines a transform matrix to be applied to the geometry.
The render state objects are ``rendered'' only nor non-shadow passes, and they are rendered
immediately before the geometry.
The render state settings are ``additive'', that is, they will affect the current OpenGL state,
but there is no pop/push mechanism for restoring the old state after rendering.
The transformation on the other hand is passed on to OpenGL by means of the OpenGL glPushMatrix/glPopMatrix
mechanism. The matrix itself should be in row-major order. (It is multiplied with the OpenGL modelview matrix
by means of glMultTransposeMatrix, to conform to OpenGL's column-major matrix order.)
The transform matrix is \emph{not} kept inside the GLShape object. Rather, it is a reference to a matrix
elsewhere, typically inside a VJoint object from hmi.animation.

\item GLBasicMesh: A GLRenderObject that defines a simple geometry ``mesh''.
It defines a number of OpenGL vertex attributes and a (shared) index structure.
The attributes are kept in objects of type GLVertexAttribute.
The vertices should be arranged by means of triangles.

\item SkinnedMesh: an extension of GLBasicMesh that can handle meshes which can be
\emph{modified} or \emph{deformed} real-time by means of a skeleton based technique.
It keeps references to several transform matrices (one for every skeleton joint), and
uses some extra vertex attributes like joint indices and joint weights.
Since it must deform the \emph{original} vertex data, it keeps copies of these original
data arrays for vertex coordinates and vertex normals.

\item GLVertexAttribute: helper class for GLBasicMesh. Basically a wrapper for
a java.nio.FloatBuffer containing the raw data for one OpenGL vertex attribute.
This data can be initialized in various ways, for instance, by copying from a hmi.scenegraph.VertexAttribute object.
GLVertexAttribute deals with the complex OpenGL attributes ``old style'', as well as with attributes
for more modern shader programming.

\item GLUtil: a basic util, for reporting GL errors etcetera.

\item Renderer4: A basic top-level ``render engine'' that deals with the obligatory Jogl setup.
So it defines the Jogl GLEventListener interface, deals with Jogl generated events
like int(), display(), displayChanged(), and reshape(). It will \emph{cause} a Jogl display call
whenever it receives a time() call, typically  from a hmi.util.SystemClock. Then, upon receiving
the (self-indiced) Jogl display call, it will start a render phase by calling the glRender() method
for a top-level GLRenderObject, that is called the ``scene''.
A few display settings can be made: the near and far plane, and the field-of-view angle.

\end{itemize}

\subsection{Package: hmi.graphics.opengl.scenegraph}

Although the opengl package can be used for ``standalone'' OpenGL demo programs,
it will typically be part of a larger setup, where the graphics is being defined using
the hmi.graphics.scenegraph package. OpenGL is not a scene graph api, therefore the scenegraph related
elements of the render engine have been placed in a special package: hmi.graphics.opengl.scenegraph.
This package mainly includes classes for the \emph{compilation} from hmi.grapgics.scenegraph structures
to hmi.graphics.opengl structures. But it also contains a few scenegraph classes: VGLNode
and VGLRenderStruct. These classes are based not only on the opengl classes, but also on hmi.animation.VJoint.

\begin{itemize}
\item VGLNode: A GLRenderObject containing a VJoint that defines a transform matrix, and a GLRenderList
containing GLRenderObjects, to be rendered with a OpenGL modelview matrix defined by the VJoint.
A VGLNode is a scenegraph node: you can \emph{add}  child VGLNodes. The effect
of an addChild operation is to create a VJoint parent-child relationship, and to
simply \emph{combine} the GLRenderLists. Of course, one would typically expect the GLRenderObjects
to refer to matrices from the VJoint tree.

\item GLScene: ``work under construction''. The idea is that VGLNodes are not quite sufficient
when complete Collada-like structures have to be dealt with.
So, we have a GLScene that should keep the ``top-level'' structures:
a list of ``root'' VJoints, a list of GLShapes, a list of SkinnedMeshes.



\end{itemize}


