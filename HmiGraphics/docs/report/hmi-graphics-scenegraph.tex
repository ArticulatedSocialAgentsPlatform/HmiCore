
\section{Scenegraphs}

The hmi.graphics.scenegraph package defines classes for modeling generic
scenegraphs, that are not tightly bound to either a graphics format like X3D or Collada,
nor to some particular render technology, such as OpenGL/Jogl.
This type of scenegraph objects is mainly used as an intermediary  between the readers
for graphics formats and the graphics render engines, based on OpenGL.
Within this section ``scenegraph'' refers to a hmi.graphics.scenegraph.


Scenegraphs are simple compared to either Collada documents or OpenGL render structures.
At the top-level we just have a \verb"GScene" element that holds a list of root nodes
for the scene graph as a whole.
The scene graph nodes are represented here by means of \verb"GNode" objects. Each of these
defines a transformation, (optionally) a list of recursive \verb"GNode" child nodes, and (optionally)
a list of \verb"GShape" nodes.
A \verb"GShape" is basically just a combination of geometry, in the form of a \verb"GMesh" object,
and ``appearance'', in the form of a \verb"GMaterial" object.

\subsubsection{Scenegraph skeletons}

 An important issue is how to represent ``skeletons'', or ``bone structures'', or whatever they are called
 at various places. We distinguish two different strategies:
 \begin{enumerate}
 \item skeletons combined with \emph{segmented} geometry.
 \item skeletons with deformable geometry.
 \end{enumerate}

The first alternative is not really special from the scenegraph point of view.
It is just that there are some rather deeply nested \verb"GNode" structures, each \verb"GNode"
representing one skeleton ``joint''. The geometry of an avatar is divided into
different ``chunks'' or `` segments'', where each segment is attached to
the appropriate \verb"GNode" joint. Each of these segments has its own transform, defined
by the cumulative effect of all \verb"GNode" transforms from the scene graph root node up to and
including the transform from the \verb"GNode" to which the segment is attached.
Because of this rather simple transform strategy, the avatar might appear to consist
of individual disconnected segments, and with certain positioning of the limbs, might show ``holes''
in the avatar's body.

The second alternative is more sophisticated, and assumes that the avatar is modeled as one seamless mesh,
rather than being broken up into discrete segments. The individual vertices of the mesh are
controlled by one or more skeleton joints, where the influence of a joint upon some
particular vertex is specified by means of a \emph{weight}.
In this case the deformation of the mesh can be calculated only when the transforms of all relevant skeleton joints
are known. An extra complication here is that meshes are usually specified in what is called the ``bind pose'',
which normally would differ from the neutral pose for animation purposes. To correct for this,
the mesh defines a number of bind pose matrices. (actually it keeps the \emph{inverses} of these matrices).

The scenegraph representation of the second alternative is as follows:
The \emph{skeleton structure} as such is represented by a hierarchically nested set of
\verb"GNode" objects. There is \emph{no} geometry attached to these \verb"GNode"s.
The avatar mesh itself is represented by a special kind of \verb"GMesh"
called a \verb"GSkinnedMesh", attached to some \verb"GShape" like any other geometry,
and combined with appropriate \verb"GMaterial".
The skinned \verb"GMesh" includes skinning information, like
vertex weights and inverse bind matrices, and including the \verb"id" of a skeleton and the \verb"sid"'s of
the joint names.
%Within a complete \verb"GScene", a resolve operation
%can be used to actually link these joint names to \verb"GNode" joints.

\subsubsection{More detailed description of scenegraph classes}
\begin{itemize}
\item GMaterial: represents material settings for graphics objects.
\item GMesh: neutral format for representing geometric mesh structures.
A Gmesh defines a number of VertexAttribute elements, each representing some
attribute like coordinates, normals, texture coordinates etcetera.
Such vertex attributes can be \emph{indexed} data structures, where each attribute
has its own index data, or all attributes can \emph{share} a single index structure,
defined only at the level of a GMesh. A GMesh can convert from individual indices to
a shared index. The latter is required for, for instance, OpenGL rendering.
A Gmesh has a \emph{mesh type}, which is one of
``Triangles'', ``Trifans'', ``Tristrips'', ``Polygons'', or ``Polylist''.
For polygons, a vcount attribute defines the number of vertices for each polygon.
A conversion from polygons to triangles is possible.
The ``standard'' usage for a GMesh is as follows:
\begin{enumerate}
\item Allocate a new Gmesh
\item Define the mesh data, by calling setIndexedVertexData for every named vertex attribute.
\item Call unifyIndices(), in order to get rid of individual attribute indices which
will be replaced by one global index.
\item Call triangulate(), if necessary, in order to get a ``Triangles'' type, rather than
``Polygons''.
\item When necessary, a call to cleanupTriangles is useful to get rid
of triangles which have a surface area below a specified threshold.
Or you can just verify that all triangles do have a specified minimal size
by calling checkTriangleIntegrity.
\item When necessay, it is possible to apply transformations or non-uniform scaling
to the mesh data.
\item Finally, you can \emph{retrieve} the mesh data: use getIndexData() to get the unified index,
and use (for instance) getVertexAttributeList() to get a Java List with all attributes.
\item An alternative route would be to start out with a unified index, rather than attributes with individual indices.    In that case, the mesh data can be defined by calling setVertexData, (for every attribute),
    and setIndexData, for the shared index data.
\end{enumerate}
A more advanced form of Gmesh defines ``special purpose'' attributes: joint names, inverse bind matrices,
and vertex weights. These are not simple VertexAttribute elements, and therefore have
their own getter/setter methods.
\item VertexAttribute
A VertexAttribute is a wrapper for data that corresponds to a single vertex data attribute, such as
vertex coordinates, vertex normals, texture coordinates, etcetera. This data can be \emph{indexed} data.
Basically, the vertex data is just a float array, and the index data is an int array.
A parameter called attributeValueSize (typically with small value like 2, 3, or 4) defines
the number of consecutive floats that define a single attribute value.

\item VertexWeights: can be seen as a special kind of VertexAttribute, used to define
the link from mesh vertices to ``joints'', used for animation purposes.
For each vertex, a vcount value defines the number of joints associated with that vertex.
The joint indices and joint weights define which joints, and with what blend weights, are to be used,
for every vertex.

\item GNode: A GNode is a generic scenegraph node. It can have (recursively) GNode types children.
It also defines a List of GShapes, associated with the scene graph node.
It also defines a local transformation, defined by a rotation, translation, scaling, and
optionally a scale/skewing matrix.
A GNode defines Collada like attributes like ``id'' (a globally unique id), ``sid'' (a scoped id, unique amongst a collection of children, and (friendly but not necessarily unique) ``name''.

\item GShape: Just a wrapper for a GMesh and a GMaterial combination.
\item GScene: The ``top level'' scenegraph element, representing a complete ``scene''.
Basically just a List of GNode typed \emph{root nodes}. So we do not necessarily have a unique root
for the complete scene, rather we have a ``forrest'' at the top level.


\end{itemize} 