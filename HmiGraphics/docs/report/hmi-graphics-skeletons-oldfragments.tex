OLD:


\subsubsection{Adjusting bind poses OLD}


Mesh data is available in various formats, usually exported from some modeling tool
in some graphics format like 3DS,  SMD, FBX, or Collada.
Here we will focus on the Collada representation.
One of the potential problems with exported mesh data is that it often mimics the
internal representation of some modeling tool. In particular, skeletons and (inverse-) bind matrices
that are present within Collada files are not necessarily the most logical ones.
Moreover, tools usually do not take the HAnim pose as the neutral pose for determining their bind matrices.
Another annoyance is that skeletons and meshes often have various conventions concerning
the so called `up axis'', and often use \emph{different} conventions for the mesh and skeleton
within the same data file.
The result is often that bind poses and/or animations include something like a ninety-degree rotation
around the X-axis for the humanoid root joint. This type of mismatch between the two coordinate systems
for the mesh and for the skeleton can be quite confusing at times.
We discuss here how to get rid of such problems by adjusting
and simplifying skeleton data and mesh data.
Our goal will be reached in a few steps:
%
\begin{enumerate}
\item First, we want to ``normalize'' the data somewhat by setting the bind pose to be the neutral pose.
 The advantage is that we can simplify our bind matrices very much,
 in that they assume that there is \emph{no} rotation or scaling at all,
and only non-trivial joint translations remain.
The latter are in fact just the vectors pointing from the
origin to the (center positions of the joints.
The main reason for this step is to get rid of complicated bind matrices introduced by design tools like 3DMax.
\item The next step is to adjust the orientation of mesh data and skeleton. We prefer an orientation
where ``Y is up''. Sticking to this convention is necessary for animation purposes, where we must assume that
poses are defined within a well defined coordinate system.
As a consequence, we apply a transform (typically something like ninety degree rotation) to the mesh data,
in order to adjust the up axis for the mesh.
\item Finally, we want to adjust the bind matrices once more.
The goal this time is to introduce a \emph{new} neutral pose, like for instance the HAnim neutral pose.
The latter is defined by a set of constraints, all formulated in a coordinate system where Y is up.
Its advantage is that it is the same pose for all virtual characters, independent of the various
bind poses that might have been used for different virtual humans.
Usually, the bind pose is something called the T-pose. But unlike the HAnim pose, this is
not a precisely defined pose, and there are subtle differences between various T-poses.
For this reason, a T-pose is less suitable as standard neutral pose.
\end{enumerate}
%
\subsubsection{Simplifying bind matrices (old version)}
For the first step we first want to justify that it is possible to switch to a set of alternate bind matrices
$B'_i$ that are simpler, while retaining the same animation capabilities.
(That is, we will have to use different $A_i$ matrices as well).
Let's assume that we have some existing bind matrices $B_i$ of the form $B_i = T_{C_i}\:U_i$, where
$C_i$ is the translation vector, and where $U_i$ is the rotation and scaling part.
A certain animation pose is defined by a set of joint matrices $A_i= T_{t_i}\: R_i$ and
an equation like \autoref{eq:weightblending} for weight blending:
\begin{equation}\label{eq:animationpose}
 \sum_{i=0}^{n} w_i\, A_i\,B_i^{-1}
 = \sum_{i=0}^{n} w_i\, T_{t_i}\:R_i\;U_i^{-1}\: T_{-C_i}
 = \sum_{i=0}^{n} w_i\, T_{t_i}\:R'_i\; T_{-C_i}
 = \sum_{i=0}^{n} w_i\, A'_i\:{B'}_i^{-1}
\end{equation}
In the last step, we have introduced new pose matrices $A'_i = T_{t_i}\,R'_i$, where the new rotation/scale matrix is $R'_i =R_i\,U_i^{-1}$,
and where we have new bind matrices $B'_i = T_{C_i}$. Clearly these bind matrices have just a translation part,
and an identity transform for the rotation and scaling part.
From this formula it is also clear that the joint center locations are defined by the $C_i$ vectors
from the bind matrices.
The conclusion is that we can switch from $B_i$ type to $B'_i$ type bind matrices by simply dropping
 the rotation and scaling part of the $B_i$  matrices.
Since it is custom to specify the $B_i$ matrices by means of \emph{inverse} bind matrices $B_i^{-1}$,
we must invert the latter  in order to get the right
translation vectors $C_i$. So  if $B_i = T_{C_i}\:U_i$ then $B_i^{-1} = U_i^{-1}\: T_{-C_i} = T_{C'_i}\:U_i^{-1}$,
where vector $C'_i$ is defined by $B_i = U_i^{-1}(-C_i)$.
(So we cannot obtain vector $C_i$ immediately from the inverse bind matrix $B_i^{-1}$ and we must invert the latter)
Now that we have our bind matrices, we ask what are the local joint matrices that will put
the avatar in its bind pose. This is simple: we know that the \emph{global} joint matrix $A_i$
must cancel the inverse bind matrix $B_i^{-1} = T_{-C_i}$, so we conclude that $A_i = T_{C_i}$.
Then the \emph{local} joint transform must be $T_{C_i - C_{\textit{parent(i)}}}$,
where $C_{\textit{parent(i)}}$ denotes the center position for the parent of joint $i$ within
the skeleton tree. (For the skeleton root joint we assume a virtual ``parent''  at the origin)
Remark: if the original inverse bind matrices include a scaling component, then this scaling
component ends up in the new pose matrices, which would be inconvenient.
If we assume that the raw mesh data, i.e without any transforms applied, is the correct
bind pose shape, including scaling of body parts, then it is no problem to omit this scaling in the
new bind matrices since the calculated joint centers $C_i$ will take the skeleton scaling into account.


\subsubsection{Bind poses}
Before we start simplifying we have a more detailed look at bind poses and bind matrices.
Let's assume that, in order to put the skeleton in the bind pose, we must apply
local joint transformations of the form $T_{t_i} L_i$, where $t_i$ is a local translation, and $L_i$ is a local
rotation (and scaling).
We consider the concatenation of such transformations along a path within the skeleton, starting at
the humanoid root, and ending in some joint $J_i$.
For convenience we assume that this chain consists of joints $J_0, J_1, J_2, \ldots, L_i$.
The  global transform for  joint $i$ for the bind pose  must be equal to $B_i$, for in that case it will be canceled by the inverse bind matrix $B_i^{-1}$.
 Therefore, we see that
 \[ T_{t_0} L_0 T_{t_1} L_1 \cdots T_{t_i} L_i = B_i = T_{C_i} U_i.\]

 With this joint pose, we have that the combination $T_{C_i} U_i U_i^{-1} T_{-C_i}$
 will cancel, as desired. If we replace the $U_i$ by something else, say by $R_i$,
 we have a transform of the form $T_{C_i} R_i U_i^{-1} T_{-C_i} = T_{C_i} R_i' T_{-C_i}$,
 which amounts to a $R_i'$ rotation around the rotation center $C_i$.
 Clearly, the translation from the bind matrix determines the joint center $C_i$.

%
The left hand side of this equation can be rewritten as $T_t L$, and from this we see that:
\begin{gather}
 L = L_0 L_1\cdots L_i = U_i. \notag \\
 t = t_0 + L_0(t_1) + \cdots + L_0 L_1\cdots L_{i-1}(t_i) \notag\\
  = t_0 + U_0(t_1) + \cdots + U_{i-1}(t_i) = C_i \notag
\end{gather}
%
Now assume that we have some arbitrary pose for the original skeleton,
based upon the original bind matrices,
specified by a series of rotations $R_0, R_1, \ldots, R_n$.
(Such data occurs in keyframe data like, for instance, in our SkeletonInterpolators)

For this pose joint $J_i$ has a matrix $M_i$ of the form:
%
\[ M_i = T_{t_0}\, R_0\, T_{t_1}\, R_1 \cdots T_{t_i}\, R_i\, U_i^{-1}\, T_{-C_i}\]
%
We would like to get rid of the rotation and scaling components $U_i$ in the bind matrices,
only leaving the translation parts $T_{-C_i}$.
Of course, the consequence is that existing pose data must be adapted.
(New pose data can of course be made according to the new bind matrices).
So we suppose that we have modified local transformation, specified by local translation vectors
$t_j'$ and local rotation/scaling matrices $R_j'$, resulting in a global matrix $M_i'$:
%
\[ M_i' = T_{t_0'}\, R_0'\, T_{t_1'}\, R_1' \cdots T_{t_i'}\, R_i'\, T_{-C_i}\]
%
We require that $M_i' = M_i$, something we
can this by making the following choice for $t_i'$ and $R_i'$:
%
\begin{gather}
t_i'= U_{i-1}(t_i)\notag\\
 R_i' = U_{i-1}\,R_i\,U_i^{-1}\notag
\end{gather}
Here, we use the convention that $U_{-1} = Id$, so $t_0' = t_0$.
In essence, we claim that the translation $t_0$ for the humanoid root need no change,
and we model this by a ``virtual'' node at level $-1$ with a bind matrix equal to the identity matrix.
Note that from the equation for $C_i$ above, we also have the equation:
\[ t_i' = U_{i-1}(t_i) = C_i - C_{i-1}\]
%
So, the modified translation vector $t_i'$ is simply the translation vector from joint $J_{i-1}$ to joint $J_i$,
within the bind pose, which seems reasonable.
But we will give a proof here:
%
\begin{gather}
 M_i' = T_{t_0'}\, R_0'\, T_{t_1'}\, R_1' \cdots T_{t_i'}\, R_i'\, T_{-C_i}\notag\\
 = T_{t_0}\,R_0\,U_0^{-1}\,T_{U_0(t_1)}\,U_0\,R_1\,U_1^{-1} \cdots U_{i-1}^{-1}\,T_{U_{i-1}(t_i)}\,U_{i-1}\,R_i\,U_{i}^{-1}T_{-C_i}\notag\\
 = T_{t_0}\,R_0\,T_{U_0^{-1}(U_0(t_1))}\,U_0^{-1}\,\,U_0\,R_1 \,U_1^{-1}
  \cdots T_{U_{i-1}^{-1}(U_{i-1}(t_i))}\, U_{i-1}^{-1}\, U_{i-1}\,R_i\,U_{i}^{-1}T_{-C_i}\notag\\
  = T_{t_0}\,R_0\,T_{t_1}\,R_1 \,U_1^{-1} \cdots T_{U_{i-1}^{-1}(U_{i-1}(t_i))}\, U_{i-1}^{-1}\, U_{i-1}\,R_i\,U_{i}^{-1}T_{-C_i}\notag\\
  = \cdots = T_{t_0}\,R_0\,T_{t_1}\,R_1 \cdots T_{t_i}\,R_i\,U_{i}^{-1}T_{-C_i}  = M_i\notag
\end{gather}
(This was to be shown.)

It is clear that the revised translations $t_i'$ are independent of the the pose, and can be
calculated from the original bind matrices.

\subsubsection{Simplifying bind matrices}
For the first step we aim at simplifying bind matrices $B_i$ such that they consist of a translation only,
i.e. without rotation or scaling.
As a consequence, existing animation data, that works in conjunction
with the original bind matrices, must be adapted.
(New animation data can of course simply ignore the old bind matrices)


Our new bind matrices are then simply the matrices $B_i' = T_{C_i}$, that is, the original bind matrices
without the rotation and scaling.

We now ask what are the local joint transformations will put
the avatar in its bind pose.
This is simple: we know that the \emph{global} joint matrix $A_i$
must cancel the inverse bind matrix $B_i^{-1} = T_{-C_i}$, so we conclude that $A_i = T_{C_i}$.
Then the \emph{local} joint transform must be $T_{t_i}$ where $t_i = C_i - C_{\textit{parent(i)}}$,
and where $C_{\textit{parent(i)}}$ denotes the center position for the parent of joint $i$ within
the skeleton tree. For the skeleton root joint we assume a virtual ``parent'', located at the origin.




\subsubsection{Transforming animations}

An extra complication arises when we have a skeleton where we have simplified the bind matrices, but where we
also have some skeleton pose (possibly part of some animation) that was constructed relative to the original
bind poses. In this section we figure out how to adapt such poses.
We assume the following:
\begin{itemize}
\item
\end{itemize}



\subsubsection{Reorienting your avatar}
The next step that we want to discuss is how to re-orient mesh data. What we want is an avatar
with its mesh and skeleton aligned with the Y-axis, looking into the positive Z-axis direction.
Slightly more general, we want to apply some affine coordinate transform $A$ to the mesh, skeleton,
and animation poses. For example, if we have some avatar with mesh and skeleton aligned with the Z-axis,
and some pose where the shoulder joint rotates $-45^\circ$ around the Y-axis,
then after the coordinate transform we have a mesh and skeleton aligned with the Y-axis, and the same
pose now has a shoulder joint rotation of $+45^\circ$ around the Z-axis.
In this case, the coordinate transform $a$ is a rotation of $-90^\circ$ around the X-axis.

Assume that we have some affine coordinate transform $A$ of the form $A=T_{t_A}\:U$, where $U$ is linear.
Informally it is already clear that we want to transform mesh coordinates $v$ into $v' =A(v)$.
The question now is: how to adapt the skeleton and animation poses.
Assume that some pose is described by local affine joint transforms $D_i$ which
describe local joint pose like the shoulder joint in the example above.
Within the new coordinates, the effect of $D_i$ will be achieved
by $D'_i = A\:D_i\:A^{-1}$.
(Just examine the effect of $D'_i$ on some ``new'' coordinate of the form $v' = A(v)$. The result is that
$D'_i(v') = A\:D_i\:A^{-1} (A(v)) = A\:D_i(v)$.)
Of course this is to be expected: $A\:D_i\:A^{-1}$ is just the standard linear algebra result
for how matrices change under coordinate transforms.
Now we must take into account that the local transforms $D_i$ that describe some pose
are affine transforms of the form $D_i = T_{t_i}\:L_i$ where $t_i$ is the (fixed) local skeleton translation
from the parent of joint $J_i$ to $J_i$ itself, and where $L_i$ is the (changing) rotation matrix for joint $J_i$.
We can rewrite the  $D'_i$:
\begin{equation}
D'_i = A\:D_i\:A^{-1} = A\:T_{t_i}\,L_i\:A^{-1}
=A\,T_{t_i}\,A^{-1}\:A\,L_i\,A^{-1}.
\end{equation}
The rightmost three factors, i.e. $A\,L_i\,A^{-1}$, are just the transformed rotations $L'_i$, adapted to
the new coordinate system. For instance, if $L_i$ would be the $-45^\circ$ around the Y-axis from the example
above, and $A$ would be the coordinate transform defined by a $-90^\circ$ rotation around the X-axis,
then $A\:L_i\:A^{-1}$ is actually the $+45^\circ$ around the Z-axis, as expected.

\noindent
The left most factors, i.e. $A\,T_{t_i}\,A^{-1}$ are the modified translations.
Because $A=T_{t_A}\:U$, we can simplify this considerably:
\begin{equation}\label{eq:transormedtranslation}
T_{t_A}\:U\:T_{t_i}\:U^{-1}\:T_{-t_A}
=T_{t_A}\:T_{U(t_i)}\:U\:U^{-1}\:T_{-t_A}
=T_{U(t_i)}
\end{equation}

\noindent
Finally, we can see how this all fits together, when we apply an affine transform $A$ to
a mesh that is being deformed by means of weight blending, specified by \autoref{eq:weightblending}.
We assume that $A = T_{t_A}\,U$, that $A_i = T_{t_i}\,L_i$, that $B_i^{-1} = C_i\,T_{-c_i}$.
We denote by $L'_i$ the transformed $L_i$, that is, $A\,L_i\,A_i^{-1}$, and we define $C'_i = A\,C_i\,A_i^{-1}$.
Then the result of applying $A$ yields the following transformed blend equation:
%
\begin{equation}
 A \:\sum_{i=0}^{n} w_i\, A_i\,B_i^{-1} = \sum_{i=0}^{n} w_i\, A\,A_i\,B_i^{-1} \quad\text{, where}\notag
 \end{equation}
 %
 \begin{gather}
 A\,A_i\,B_i^{-1} = \notag\\
 A\, T_{t_0}\,L_0\,T_{t_1}\,L_1\cdots T_{t_i}\,L_i,C_i\,T_{-c_i}=\notag\\
 A\,T_{t_0}\,A^{-1}\,A\,L_0\,A^{-1}\,A\,T_{t_1}\,L_1\cdots T_{t_i}\,L_i\,C_i\,T_{-c_i} =\notag\\
 T_{U(t_0)}\,L'_0\, A T_{t_1}\,L_1\cdots T_{t_i}\,L_i\,C_i\,T_{-c_i} =\cdots=\notag\\
 T_{U(t_0)}\,L'_0\,T_{U(t_1)}\,L'_1\cdots T_{U(t_i)}\,L'_i\,C'_i\,T_{U(-c_i)}\,A\notag
 \end{gather}
 %
 The remaining $A$ at the right of this last formula is the transform to be applied on the mesh.
 Since the translations, and the bind matrices are fixed, they can be calculated before we start rendering.
 Finally, the transformed joint rotations are most easily obtained by creating animations
 in the ``new'' coordinate system right away, rather than by  transforming
 animations that were created using the ``old'' coordinates.
 Note that $L_i'$ will be an affine transform, unless $A$ itself is a simple linear transform.
 In that case, $T_a=0$, and $L'_i = U\,L_i\,U^{-1}$.

\subsubsection{Redefining the avatar neutral pose}

The result of the previous sections is an avatar where the original bind pose is also the neutral pose,
that is, when all local rotation matrices are set to identity, the avatar will assume a pose equal to
the bind pose. We would like to change this, and define some other pose, say the HAnim pose,
to be the neutral pose.
This problem can be split into two steps:
\begin{enumerate}
\item First, find out how to put the avatar in the HAnim pose,
\item Second, define that pose as the neutral pose, by adapting transformations and bind matrices.
\end{enumerate}
%
We start with the second step, so, we assume that we have a a pose defined by local transforms $L_i$
that define a pose that we would like to set as the neutral pose.
The (current) transform for joint $J_i$ for this pose is:
%
\begin{equation}
T(t_0)\,L_0\,T(t_1)\,L_1\,T(t_2)\, L_2\cdots T(t_i)\,L_i\,B_i^{-1}\notag
\end{equation}
%
This we would like to rewrite such that the $L_i$ rotations become \emph{identities}:
\begin{gather}
T(t_0)\,T(L_0(t_1))\,L_0\,L_1\,T(t_2)\, L_2\cdots T(t_i)\,L_i\,B_i^{-1} =\notag\\
T(t_0)\,T(L_0(t_1))\,T(L_0 L_1(t_2))\,L_0\,L_1\, L_2\cdots T(t_i)\,L_i\,B_i^{-1} =\notag\\
T(t_0)\,T(L_0(t_1))\,T(L_0 L_1(t_2))\cdots T(L_0 L_1\cdots L_{i-1}(t_i) \:L_0\,L_1\,L_2\cdots L_i\,B_i^{-1} =\notag
\end{gather}
%
From this equation it is clear how to proceed:
\begin{enumerate}
\item The local translation vectors $t_i$ have to be replaced by $L_0 L_1\cdots L_{i-1}(t_i)$,
\item The inverse bind matrix $B_i^{-1}$ must be replaced by $L_0\,L_1\,L_2\cdots L_i\,B_i^{-1}$.
\end{enumerate}
