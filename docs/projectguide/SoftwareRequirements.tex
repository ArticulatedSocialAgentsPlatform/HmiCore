
\chapter{Software Requirements}\label{chapter:softwarerequirements}

Our HMI projects are based on tools and languages including the following ones:

\begin{enumerate}
\item An \textbf{SVN client}, like Tortoise SVN or SmartSVN. There is even a commandline svn.
\item The \textbf{Java development kit}, also called the JDK.
\begin{itemize}
\item There are many variations of Java on \verb#java.sun.com#, but what
you need is the Java Standard Edition, also called Java SE.
\item If you have some version installed already, check the version from the command line with
\begin{verbatim}
    java -version
    javac -version
\end{verbatim}
Both should report some 1.7.x version or later.
\item If \verb#java# or \verb#javac# does not work from the command line,
ensure that Java is installed, and that the java/bin directory is present on your executable path.//
For Windows-XP this is probably the \verb"C:\Program Files\Java\jdk1.6.x\bin" directory,
for Windows 7 64--bit it is typically \verb"C:\Program Files (x86)\Java\jdk1.6.x".
Note: Even on 64--bit systems, you can still use the 32-bit version of Java.
\end{itemize}
\item    You can compile and run with the standard \verb#javac# and \verb#java# commands,
    but it is easier to do this using the \textbf{ant} build tool.
    Our Hmi project style even requires an \verb"ant" build file for all projects.
    \begin{itemize}
    \item Check: From the command line you can verify that \verb#ant# is installed with:\\[1ex]
    \verb#ant -version# \\[1ex]
    It should say something like "Apache Ant version 1.8.2 "
    If it is not installed, or you have a version older than 1.8.2:
    \item Download \verb#ant# from \verb#ant.apache.org#.
    \item ``Installation'' of \verb#ant#: just unzip, and ensure that the ant/bin directory is on your executable path.
    \item For large (re)builds, especially of HmiDemo projects with lots of dependencies, the default \verb#ant# settings do not suffice, and \verb#ant# might crash because of memory problems.
        When you run \verb#ant# from the (Windows) commandline, you run \verb#ant.bat# from the \verb#ant/bin# directory. That file reads an optional settings file called \verb#antrc_pre.bat#.
        It searches for this file in several places, including in your HOME directory or your USERPROFILE
        (Like ``C:/Documents and Settings/username'').
        The ``out-of-memory'' problems can be solved by creating this file with a line like:\\
        \verb#set ANT_OPTS=-Xmx256M -XX:MaxPermSize=256M#.
    \end{itemize}
 \item \verb"ant" is good for basic building, and in combination with a good editor like  \textbf{UltraEdit},
 it can act as a basic development system.
 A good alternative is to use some development tool like \textbf{Eclipse} or \textbf{Netbeans}.
 \item \textbf{Netbeans} is a comprehensive development tool, freely available from
 \verb#www.netbeans.org#. The simplest option is to download the ``complete'' version, and to install
 only those modules you need. There are more plugins on the web, for instance.
 Netbeans is based on \verb#ant# build scripts, which can be very convenient, but
 also causes some problems for projects that need to cooperate with others not using Netbeans.
 See the chapter on Netbeans.
 \item \textbf{Eclipse} Popular alternative alternative for Netbeans.
 \item \textbf{UltraEdit} is a good general purpose editor. EWI has this software available.
 (It is \emph{not} free). With proper configuration, it can be used as a ``mini'' development tool.
 \end{enumerate}





