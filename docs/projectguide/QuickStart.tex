\chapter{QuickStart}\label{chapter:quickstart}

We assume that the following software is installed: \verb"Java", \verb"ant", and some \verb"svn" client.
If something is missing here, see  chapter \ref{chapter:softwarerequirements} on software requirements.
\begin{enumerate}
\item Allocate some directory for your Java projects. It will act as the base directory where
 directories for shared repositories will reside. It is also a convenient ``root directory'' for your own projects.
For the rest of this document we will call it ``\verb"<JavaProjects>"''.
Example: For XP or Win-dows 7, \verb"<JavaProjects>" would typically be a directory like  \verb"C:\JavaProjects".
\item Use your favorite \verb"svn" client to checkout the \verb"HmiShared" directory from our HMI SVN repositories.
It is found in the
\verb"http://hmisvn.ewi.utwente.nl/hmishared" repository.\\
You should end up with the \verb"HmiShared" directory as subdirectory of \verb"<JavaProjects>".
Within \verb"HmiShared", you will find our common \verb"ant" build scripts. A second important part of \verb"HmiShared" is a
\verb"repository" directory containing library files, mostly \verb"jar" files, that are used in most projects.

\item To check that it all ``works'' you might want to run one of the example projects, like our  \verb"HmiGraphicsDemo":
\begin{itemize}
\item Go to a dos-prompt in the root of the (example) project.
\item     First run \verb"ant resolve" This will retrieve the necessary library files
from various repository directories like \verb"HmiShared/repository" or \verb"Shared/repository",
and puts these library files into your project's \verb"lib" directory.

\item     Finally run  \verb"ant". This will compile, then  run the main class of the project.
\end{itemize}


\item You might then check out some selected directories from
the \verb"Shared" directory.
This repository contains, for example, data files for several specialized avatars and sound files for the Elckerlyc project.
It is available from  the same HMI SVN repository mentioned above.
\item It is advisable to check out the \verb"HmiDemo" modules, because they contain several useful code examples for many purposes. The \verb"HmiDemo" module is found in the \verb"http://hmisvn.ewi.utwente.nl/hmidemo" SVN repository.
\item Now you can create a new Java project, or you can study one of the ``demo'' projects.
After checking out, you might have a directory structure that looks like this:
\begin{verbatim}
<JavaProjects>/HmiShared
<JavaProjects>/Shared
<JavaProjects>/HmiDemo
<JavaProjects>/ExampleStudentProject
<JavaProjects>/MyOwnNewProject
\end{verbatim}


-------checkout Project non-recursively.


\item You can create a project, for instance by copying material from the  \verb"ExampleStudentProject".
 demo. The \verb"build.xml" file in the root of your newly created/copied project should be ok ``as is''.
 On the other hand, the \verb"build.properties" file might need some editing.
 For instance, if your project needs library files from other projects, you can specify this
 in \verb"build.properties".
 See section \ref{XX} below for details.

 \item When you have created a new project yourselves, you can share it with others by uploading it to
the \verb"http://hmisvn.ewi.utwente.nl/hmistudent" SVN repository. (See section \ref{xyz} for more detail how to do this.)

\item We have a useful WIKI: http://asap-project.ewi.utwente.nl/wiki/BuildSystem


\end{enumerate}

