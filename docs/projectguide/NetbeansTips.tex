\section{Netbeans tips}

We discuss some hints for Netbeans, in particular for version 7.2.

\subsection{Integrating NetBeans with our build system}

NetBeans has, in the background, an \verb#ant# based build system that is not unlike our own
build system. You can combine these, but you need to take some precautions when you create a NetBeans project.
Potential problem: by default, NetBeans want to create its own build.xml \verb#ant# file in the
project directory.
The solution is first create a project, including the appropriate build file, and the appropriate directory
structure including the \verb#lib#, the \verb#src#, \verb#test\lib#, and \verb#test\src# directories.
Check that it compiles, etcetera, but after that: do an \verb#ant clean#.
(Otherwise, NetBeans will complain that there is already a build directory in your project.)

Then there are basically two options to proceed: the first is to create a
NetBeans ``free form'' project.
In that case, Netbeans will simply pick up your existing \verb#build.xml# file, and use that
for it own builds. Disadvantage: it's ok for basic editing, compiling, etcetera,
but you cannot profit from all Netbeans features.
The second option is to create a new NetBeans ``Java project from existing sources''.
In that case, NetBeans has its own build file, now called \verb#nbbuild.xml#.
Advantage: you can profit from all netBeans features, and it will not disturb the shared
build files. Disadvantage: you must setup the NetBeans project carefully, as we will describe now.

\begin{enumerate}
\item We assume that you are using NetBeans 7.2 (or higher)
\item create, outside NetBeans, a basic project in line with our build system.
If you have an existing project, that's ok as well.
Ensure that you have ``resolved'' the project dependencies, so that the contents of
the \verb#lib# and \verb#test\lib# directories is what you need.
(Use \verb#ant resolve#)
\item Ensure that the your project has no \verb#build# directory: for instance use \verb#ant clean#.
\item Now create a new project within NetBeans. Choose ``New Java Project with Existing Sources''.
\item Choose the correct project name (by default we pick the project directory name), and select the correct
project folder. Enable the ``Use dedicated Folder for Storing Libraries'' option,
but choose a \emph{different} name than ``\verb#.\lib#''.
For instance, choose ``\verb#nbproject\lib#''.
NetBeans stores some of its ``own'' info in  its ``\verb#lib#'' folder.
 Unfortunately, our version manager Ivy will discard such info from \verb#.\lib# when you
do an ``\verb#ant resolve#'' operation, so we must give the NetBeans lib a special place.
\item ``Next'', add the \verb#src# folder to the list of ``Source Package Folders'',
and add the \verb#test\src# folder to ``Test Package Folders''.
\item Now you can ``Finish'', and you have a NetBeans project.
Except that it won't compile, because you must correct the library settings:
\item The NetBeans ``Projects'' panel on the left show two sets of libraries: ``\verb#Libraries#''
and ``\verb#Test Libraries#''
\begin{itemize}
\item Right click on ``\verb#Libraries#'' and choose ``Add JAR/Folder''. Navigate to the project's
\verb#lib# folder and add all \verb#jars# that you find there.
Also add the project's \verb#resource# directory here. (This will ensure that this directory
is on Java's classpath, so the ``getResourceAsStream'' method for loading resource data will work)
Leave the ``Reference as Relative Path''
option enabled: this prevents NetBeans from copying into its own \verb#nbproject\lib# directory.
Do the same with ``\verb#Test Libraries#'', this time using the \verb#jars# from \verb#test\lib#,
and the \verb#test\resource# directory.
As you can see we now have a \verb#junit-4.8.2.jar#, that is present in our \verb#lib# directory,
but also two more NetBeans ``libraries'', one for \verb#JUnit3.8.2# and one for \verb#JUnit4.8.2#.
The latter two should be \emph{removed}.
\end{itemize}

\item By now, the project should be in shape: no unresolved libraries, it compiles and runs,
and JUnits test work.
\item Note that whenever you change the contents of the \verb#lib# directories, you will have to
go back to the project properties page (from the File menu), and reestablish the correct
libraries settings.
For example, say you had some file ``\verb#guava-r06.jar#'' inside your \verb#lib# directory,
but after the resolve operation it has been replaced by ``\verb#guava-r07.jar#''
NetBeans now automatically removes \verb#guava-r06.jar# from its libraries, but does not
auto-understand that it needs \verb#guava-r07.jar# instead.
Right-click on the Netbeans project name, and choose ``Resolve Reference Problems...''.
Select \verb#guava-r06.jar# and click "Resolve". Now you can navigate to the project's \verb#lib#
directory, and select the new \verb#guava-r07.jar#.
For the time being, NetBeans is now satisfied. In reality, it has not really removed the reference
to \verb#guava-r06.jar#, but it has added a redirection inside your
own ``private'' netbeans settings. So next time that you resolve, it cannot find \verb#guava-r06.jar#
again. You can correct this by first removing the  \verb#guava# jar from the NetBeans library,
then adding the correct version. problem: you \emph{cannot} remove it while the reference problem is not solved.
In the end, the easy solution is to first remove \emph{all} jar files from \verb#lib# ( and also from \verb#test\lib#
if you expect changes over there) before the resolve, then add them back after the resolve.


\end{enumerate}



\subsection{ building jar files}
\begin{itemize}
\item By default, compiled class files are included in the jar file
produced by a build. But also \emph{all} files inside the src directory, except for
.java and .form files are included. That is convenient if you want to include ``resource'' files,
like data files, images , icons etcetera: put them in the appropriate source directory, and
use the Java getResourceAsStream() method for reading the data.
Although this works, our strong preference is to include ``resources'' into its own \verb#resource# directory within
your project. (Our \verb#ant# build system puts your project's own \verb#resource# directory on the classpath, as well as
other directories that you specify in the \verb#build.properties# file)
Netbeans auto-include of src documents is also not so nice if you have some secret-remarks.txt, or some huge remarks.doc
word file in your source directories. In this case, you must set the files to \emph{exclude}
for packaging. For instance, via:\\
\verb#Build | Set Main Project Configuration | Customize...#,\\
or via right-clicking the project tab, and selecting ``Properties''.
Choose the ``Packaging'' tab, and to the ``Exclude from jar file'', for instance,
add \verb#,**/*.txt, **/*.doc#.
\end{itemize}

\subsection{subversion}
\begin{itemize}
\item First \emph{checkout of an existing project}: Use the ``versioning'', and select the checkout
options (for svn). The repository name depends on the project, but will typically be something
like ``http://hmisvn.ewi.utwente.nl/hmi'', where the last ``hmi'' identifies the specific repository
from all svn repositories managed by Hmi. Use your own (EWI) username and password.
After clicking ``next'' you can \emph{browse} the repository. Select the folder you want to have.
This will typically be a folder containing a complete ``project'' like, for instance,
the Hmi/HmiAnimation folder. The commandline svn allows for specifying the local directory name;
in Netbeans, you can only select the local directory where the folder will be placed.
In the example, if you select a local directory called ``javaprojects'', then a subdirectory called
`` HmiAnimation'' (i.e. without the HMI part) will be placed in that directory.
\item By default, netbeans wants to put everything on a subversion repository except for the private
parts of the nbproject directory. We don't want the nbproject directory at all, since it sharing
it doesn't work too well, and is the cause for a lot of annoyances.
\end{itemize}


%\item The combination of a ``shared'' \verb#ant# build file and a Netbeans project poses
%some problems. It is very convenient to let Netbeans generate your \verb#ant# files, but
%without precautions, these \verb#ant# files cannot be used by anyone else.
%Even people running the same version of Netbeans will face some annoyances,
%like broken library paths.
%And if you \emph{don't} have Netbeans installed at all,
%then even the ``default'' build target fails.
%Proposed solutions:
%\begin{enumerate}
%\item The first solution is to create or generate appropriate \verb#ant# build files
%outside Netbeans and then start a new Netbeans project ``with existing ant script''.
%If your \verb#ant# file is ok, then usually Netbeans will find it, and even propose \verb#ant# targets to be used for standard Netbeans actions like compiling, running etcetera.
%You can specify this when creating the Netbeans project,
%or later on, by right-clcking the project, and then going to the ``Build and Run'' tab of the project properties.
%The \emph{standard} IDE actions are: ``Build Project'', ``Clean Project'', ``Generate Javadoc'', ``Run Project'', and ``Test Project''.
%Moreover, you can add ``Custom Contextual Menu Items''. This allows to specify labels for your own special \verb#ant# targets.
%These labels will show up when you right-click the project.
%\item The second option is to let Netbeans generate its own \verb#ant# scripts, but to
%modify these, such that you can run it independently.
%The advantage is that Netbeans scripts include somewhat complicated targets, for instance, for testing and debugging,
%that your own \verb#ant# script doesn't have.
%Moreover, it is easy to add other projects or libraries to your project
%\end{enumerate}
%For both options, you need to create some \verb#ant# scripts.
%The idea is to split this into two parts: one part for basic tasks like compiling and running, that could replace the Netbeans targets, and a second part with targets that are missing anyway from Netbeans scripts,
%or that need to be corrected.
%Think of targets for creating a ``release'' version (not what Netbeans calls a ``distribution'' version).
%The potential problem with this approach is that the Netbeans targets and your own targets are not fully consistent.
%Say, you add some library in Netbeans, but you don't update your own ant file.
%For a while it seemed that the Netbeans generated build files could be used `` as is'' by others, but that turned out to be not true.
%In order to test this, it is not sufficient to run \verb#ant# from the command line:
%it is even necessary to temporarily rename the ``Program Files/Netbeans x.y'' directory and the ``Documents and Settings/user/.netbeans/x.y'' directory.
%It turns out that in this case the ``default'' target will fail to build. Solution: rename the default target in Netbeans's  build.xml
%file, into, for instance, ``run'', or ``compile''.

%\noindent
%Some build.xml file types:
%%
%\begin{enumerate}
%\item ``Standalone'' ant build files.
%\item Netbeans generated build files.
%\item build files that were standalone, but have been used to create
%a netbeans project ``with existing ant file''.
%\item Netbeans generated build files, adapted later on to include missing features,
%or that are also suitable as ``standalone'' ant build files.
%\end{enumerate}
%%
%Typically, you start with type $1$ or type $2$, and then later on you move to type $3$ or $4$.
%The path from type $1$ to $3$ seems attractive, but not all Netbeans features will be enabled.
%Basically, with this approach, Netbeans is used as a nice editor, but not much more.
%The alternative, from $2$ to $4$ is potentially better, but also more complicated,
%it requires more planning, and usually requires some manual adjustment.
%The biggest problem with this approach is how to get \emph{consistent} settings for Netbeans and setings for ``standalone'' ant mode. The problem is that modifying settings from
%within Netbeans will result in updates to files like, for instance, the project.properties file
%inside the nbproject directory or, worse, properties files inside your private/personal
%Windows directories. Such property files are likely to be missing on other systems, or else
%they might have different and inconsistent settings.
%The alternative is to override Netbeans ant property setting in a customization properties file.
%This will guarantee consistent and portable setting, but effectively disables
%the possibility to set properties from inside Netbeans and, moreover, may give the false impression
%that you still \emph{can} control such properties from inside. 