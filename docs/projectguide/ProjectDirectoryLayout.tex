\section{Project directory layout}

We describe the preferred project directory structure. When using tools
like Netbeans or Eclipse, the tool might generate this structure, at least partially,
whenever you start a new project.
We assume that all project related files and directories are combined inside a single
directory, that we will call the ``project directory''.
The convention is to put this directory as a whole inside some \verb#svn# repository.
The location of the project directory on your own system is up to you.
We assume however that there is a \emph{shared project directory}, containing shared ant scripts,
shared libraries, project directories of one or more of your projects, etcetera.

\emph{Inside} the project directory, we assume a fixed directory structure, described below.
\begin{itemize}
\item a \verb#src# directory, containing source code.
\item a \verb#test# directory, containing (JUnit) tests.
\item a \verb#build# directory, containing compiled classes.
\item a \verb#dist# directory, containing packages build outputs.
\item a \verb#docs# directory, containing documentation.
\item a \verb#lib# directory, containing ``libraries'', i.e. jar files and things like \verb#dll# files.
\item a \verb#resource# directory, containing (fixed) data, like images, icons, other data files.

\item a \verb#build.xml# file, containing an Ant build script.
\item a \verb#manifest.mf# file, containing the manifest file for packaging into jar files.
\end{itemize}

We describe these files and directories in some more detail:
\begin{itemize}
\item The \verb#src# directory. This contains a directory structure, conform your Java package
structure, containing java source files.
It is possible, in principle, to include ``resource files'', containing fixed data like
icons, etcetera, also inside source directories. An alternative place, for such data is
inside a separate ``resource'' directory.
\item The \verb#test# directory. This is a directory like \verb#src#, but containing source code
for tests, including JUnit tests. Netbeans will generate such JUnit tests inside this directory.
\item The \verb#build# directory. This directory contains compiled classes, typically inside
a \verb#classes# subdirectory, and possibly other generated stuff. The \verb#build# directory
can be removed and regenerated at any time, for instance, by a ``clean-and-(re)build'' operation
of your development tool, or by executing the \verb#clean# and \verb#build# targets from the \verb#ant#
script.
\item The \verb#dist# directory. ``Building'' a project means more than just recompile everything:
it also includes creating a jar file containing classes, and possibly ``resources'' (see below).
This packaged form is what would be used in other projects, or could be used for a Java webstart
version of a project, or for a ``demo'' version of an application.
Mainly because of the Netbeans habit to recreate the \verb#dist# directory every time you rebuild,
this directory is not the appropriate place for a stable version that other projects should use.
(Therefore, we have a separate repository directory.)

\end{itemize} 

\subsection{Sharing, and the repository}
Project directories are in principle ``self contained'', that is, they contain all source code, resources and data,
and library files, that are needed to build and run the project. 
But off course, many projects will have dependencies on other projects, either from HMI or from elsewhere.
They may need library files or access to shared data. 
Typical example: projects \verb#HmiUtil# and \verb#HmiXml# contain utilities used by many other projects. 
Each project \verb#P# \emph{using} those utilities will have the two jar file \verb#HmiUtil.jar# and
\verb#HmiXml.jar# inside the \verb#lib# directory of  \verb#P#. 
A second typical example: we have data like Collada files for avatars, texture files, audio data, used by many projects. 
First question: how to obtain these libraries or data.
Second relevant question: what to do when
some bug fixes or even a new release is produced for some library or data set.
Our approach is that we have a \emph{shared repository} containing versions, including the ``latest beta version''
and the ``latest stable version'', of projects, including external ``projects''. 
A project that depends on some other projects can \emph{declare} this in its (\verb#ant#) \verb#build.properties# file
 and use the \verb#ant# \emph{resolve} target to copy the appropriate version to its own project's \verb#lib#
 or \verb#resource# directories. 
 The \verb#ant# build scripts make some assumptions concerning the location and directory layout
 of the repository:
 \begin{enumerate}
 \item There is a directory called the ``shared project director''.
 It doesn't really matter where it resides on your system, but a good choice is the
 parent directory of all your projects. 

 \emph{The location of the shared project directory must be specified inside each project's }\verb#ant build.xml# \emph{file}.
 \item The shared project directory contains the repository directory.
 \item The shared project directory also contains \emph{shared} \verb#ant# directory, containing shared scripts, templates and customization, used by the default \emph{project} \verb#ant# scripts. 
 \end{enumerate}