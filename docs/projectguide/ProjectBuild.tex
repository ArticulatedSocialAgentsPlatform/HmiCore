

\chapter{Hmi Java Projects}\label{chapter:projectbuilding}


This chapter describes in more detail how to deal with the Hmi Java Projects system.

\section{Philosophy}
It is often  necessary to (re)build Java projects, even by persons who
didn't develop that project. Since many different tools are being used,
we have to agree on a number of basic conventions and procedures.
\begin{itemize}
\item We do \emph{not} assume that everyone will be using the same development toolkit or any toolkit at all.
But we \emph{do} require that a project created and developed, say, by means of Netbeans or Eclipse,
can be (re)build without any of these tools available.
\item Since we have to use some common build tool, we have agreed to use \emph{ant} as
that ``common'' minimal platform.

\item In principle every project lives in its own directory, containing all relevant source code, test code,
project specific data, library files needed by that project, project documentation, etcetera.
We have a preferred standard layout for the directory structure inside a project.
A project also contains an \verb#ant# buildfile (\verb#build.xml#) and usually also a \verb#build.properties# file.
We require that we can (re)build and run a project using \verb#ant#, without any reliance on development tools
that might have been used to develop the project. To be clear: we don't want to install JBuilder or Netbeans or Eclipse or whatever just to build and run your project.
The \verb#ant# file can be either a simple ``standalone'' build file, but the preferred way is to use
a very small build file that just links to our shared build file. (See below)

\item We have a limited number of shared projects, all of which are available from our GIT repositories:
\begin{itemize}
\item There is a ``project'' called \verb#hmibuild#. It contains the shared \verb#ant# build files. 
You must have this project in order to use our build system.

\item Shared software is available as source code or as compiled jar files. Most projects use the precompiled
jar files, which are kept in a project's \verb#lib# directory. (For tools like Eclipse or Netbeans you must do some configuring
in order to use these library files, see below).
We use a tool called \verb#ivy#, used by our build files, for easy version management of \verb#lib# files 
that relies on our web  repository. (\verb#hmirepo.ewi.utwente.nl#).
Basically, when you type ``\verb#ant resolve#'', then \verb#ivy# will copy the library files
    needed for your project into the \verb#lib# directory of your project.  What will be copied is derived from
    a project file called ``\verb#ivy.xml#''. 
\item There are a few ``projects'', like \verb#HmiResource#, that contains just ``resource'' data of all sorts, that is shared
between projects. For instance, BML scripts, data for 3D scenes and avatars etcetera lives here.
Usually, you can obtain such data also from the web repository, in packaged jar format. Sometimes, you want
to actually see and modify that data, and in that case you will need to check out the relevant resource data
from the git repositories. 

\end{itemize}


\item Projects can build ``on top'' of other projects, including external projects, in a hierarchical way.
Projects that become \emph{mutually or circularly dependent} should be refactored, for instance by merging
 mutually dependent projects together into a single project.

\item Projects \emph{import and export} class code and data in the form of \verb"jar" files. There is
no sharing of \emph{source} code. This ensures that every project can be built stand alone, after importing
the necessary library files.

\item Import and export of class code and data is effectuated via shared repositories.
On your local system, this is just some directory shared by various projects.
Parts of these shared directories are also mirrored on our shared SVN repositories.


\end{itemize}



\section{Ivy dependencies}

\subsection{Why dependency management}
The contents of lib directories consists of \verb"jar" files and/or ``\verb"dll"'' or ``\verb"so"'' files
that are necessary for compiling and running the project. The basic strategy is that interdependencies between projects
are via import/export of library \verb"jar" files, in preference over direct source code dependencies.
Of course we have to face the problem of project \emph{versioning}. We have stable \emph{release} versions
of projects, but also less stable \emph{beta} and \emph{alpha} versions.
here we have some conventions and rules. For instance, we do not want a stable release version of project X
to be dependent on an unstable alpha version of project Y. The other way around, so an alpha version of X dependent
on a release version of Y is OK of course.  It will be clear that manual version management is not a good idea: lot's of work, and error prone. Instead, we use a dependency manager, Ivy.
\subsection{How does it work}
Every project has an ``\verb"ivy.xml"'' file that describes project dependencies.
The \verb"ant" build files, relying on the Ivy system, use these so called ``ivy's'' for
\emph{resolving} the contents of the project's \verb"lib" directory.
The system is based upon the notion of \emph{configurations}, \emph{versions}, and \emph{status of versions}.
For example, our build system uses ``alpha'', ``beta'', and ``release'' as possible status of some
\verb"jar" version. The ordering of these statuses is relevant. For instance, when you ask for a ``latest beta'' version
of some module Ivy will choose the latest amongst published beta and release (but not alpha) versions of that module.
Similarly, if you ask for a ``latest alpha'', anything is acceptable, but if you ask for a ``latest release'', then
only versions classified as ``release'' are taken into consideration.
Ivy allows for several ``configurations'' of the project, each with its own set of dependencies.
Currently, we have configurations for producing alpha, beta,  and release versions of the project itself, and
a ``test'' configuration for (extra) dependencies needed for running tests.
(For technical reasons we have two more configurations, called ``master'' and ``default'', which are discussed below)
The work flow is roughly as follows: first move the project into either the alpha, beta or release configuration,
second do your development, including testing etc, third publish an alpha, beta or normal release based upon the current
configuration. When you actually publish, we attach a version number to the jar file. Also,
the Ivy system records metadata concerning published modules based upon version numbers, to be used later on,
when resolving for other projects. It does so by publishing not just a \verb"jar"" file (or other ``artefacts'', as they are called by Ivy), but also an accompanying \verb"ivy.xml" file, derived from the project's ivy at the moment of publication.
In this way, the Ivy system knows not only about \emph{direct} dependencies, but is also capable
of resolving \emph{recursive} dependencies. This means that, for instance, when my module X declares (just) a dependency
on the \verb"HmiGraphics" package, the resolve process will look into the dependencies of \verb"HmiGraphics".
The result is that project X will receive \verb"jar" files for \verb"HmiGraphics", but also
for \verb"HmiAnimation", \verb"HmiXml", \verb"HmiMath", and \verb"HmiUtil", because of (recursive) dependencies.
When some \verb"jar" file is required more than once, say via a direct dependency and also via some indirect dependency,
and the versions required are not consistent, then Ivy delivers the ``latest'' version.
So, for instance, in the example above, if project X declares (direct) dependencies on the alpha version of \verb"HmiGraphics" and the release version of \verb"HmiXml", while the alpha version of \verb"HmiGraphics"
declares itself a dependency on the beta version of \verb"HmiXml", then project X will receive that beta version
of \verb"HmiXml". That should be ok: since we are asking for some alpha version (of \verb"HmiGraphics",
we should allow other alpha and/or beta versions if \verb"HmiGraphics" actually needs them, even if our own project
would be satisfied with the most recent release version.

\subsection{Version numbering}

We use some \emph{conventions}; the Ivy system does not require some particular numbering scheme, but
uses version numbers to determine which version is ``more recent''.
This includes a few subtle cases. For instance, Ivy knows that version \verb"1.4-alpha" is before
\verb"1.4", and that versions \verb"2.0-rc1" and \verb"2.0-dev384" are both \emph{before} \verb"2.0".
Our conventions:
\begin{itemize}
\item Basically, we use version numbers of the form \verb"major.minor"
where \verb"major" are  \verb"minor" are numbers.
Examples: \verb"0.1", \verb"1.4", \verb"2.0", \verb"2.0.1".
\item We allow suffixes of the form \verb"-alpha", to denote alpha release versions,
and \verb"-rci" or \verb"-devi", where \verb"i" some number, to denote ``release candidates'' and ``developer versions''.
The latter are beta releases, intended to become a full release somewhere in the future. So, for instance, \verb"2.0-dev384"
is a developer version, working towards full release \verb"2.0". 
\item The \verb"major.minor" form is the preferred form for both betas and full releases.
\item The \verb"major.minor-alpha" form is the preferred form for alpha releases.
\item The \verb"major.minor-rci" form can be used as a ``release candidate'' for release \verb"major.minor".
\item The \verb"major.minor-devi" form is used as a ``developer version'' for release \verb"major.minor".
Such versions are produced during our ``nightly'' build process. (Should be running on a daily basis; but when testing during
the nightly build produces error, no new versions are published that night)
\item Version numbers for full releases and beta releases are to be unique.
Once published, you cannot reuse the same version number for a new release anymore.
For alpha versions this would be counterproductive, since alpha versions can be produced in rapid succession,
with only very minor differences. For this reason, our strategy is that alpha releases will not be put on
the repository, so are not shared, therefore do not need unique version numbers.
\item Sometimes we use version numbers of the form \verb"major.minor.maintenance".
This form is intended for releases that are bug fixes for normal \verb"major.minor" releases.
Say currently we have a release version \verb"2.3", and we already produced newer beta versions
\verb"2.4" and \verb"2.5".
Then we detect an annoying bug in \verb"2.3". Now we could correct this and then produce
 a full release \verb"2.6", but that would also include our new/experimental/buggy beta and alpha code from \verb"2.4", \verb"2.5".
  The better solution here would be to use \verb"git" to temporarily revert
to the code of version \verb"2.3", do the bug fixing, then publish that as
\emph{maintenance} release version \verb"2.3.1".
Note that this is a \emph{release} version, not an alpha or beta, so it that will become the preferred one when resolving 
 for a ``latest release'' version.
\end{itemize}



\subsection{Ant targets for producing releases}

The ``current'' version number is kept within the project's \verb"manifest.mf" file.
It is automatically updated by some of our \verb"ant" targets, and used for Ivy publishing.

\begin{itemize}
\item \verb"ant release"  increments the minor version number, then produces a  version with ``release'' status.
When the current version happens to be a release candidate or developer version, then ``incrementing the minor''
actually means ``stripping off'' the \verb"-devi" or \verb"-rci" suffix.
\item \verb"ant majorrelease"  increments the major version number, resetting the minor number to $0$, then produces a release version. As before, when the current version happens to an developer or release candidate, then is replaced by
    ``stripping off'' the \verb"-devi" or \verb"-rci" suffix. The released version gets ``release'' status.

\item \verb"ant maintenancerelease" appends a maintenance number,or increments it when the current version already has
a maintenance number, then produces a ``release'' version. Note that this does not take care of the process of ``reverting'' to the desired release version, 
nor does it (re)set the current manifest version number to the desired one.
Currently this has to be done manually.

\item \verb"ant minorreleasecandidate" moves to the next minor version, appends the \verb"-rc1" suffix, and
produces a beta release. When the current version is already a ``release candidate, the ``rc'' number is increased.

\item \verb"ant majorreleasecandidate" is similar, but first increments the major number, as usual for a major release.
\end{itemize}




\section{Extended example}

We describe an example of how to build versions of a number of dependent packages.
We use here the HMI packages as an example.

\begin{itemize}
\item We assume that we start with existing source code, but no repository.

\item For \verb"HmiUtil":
\begin{itemize}
\item \verb"ant resolve"
This will resolve the project dependencies. For \verb"HmiUtil" this is just
a dependency on the JUnit framework, for testing.
The \verb"ivy" file for \verb"HmiUtil"
contains the line:
\begin{verbatim}
<dependency org="junit" name="junit" conf="test" rev="latest.release"/>
\end{verbatim}
This line denotes that the \verb"HmiUtil" project depends on the \verb"junit" module from
organization \verb"junit". We want the ``latest release''.
\item \verb"ant"

\item \verb"ant test"

\end{itemize}

\end{itemize}



\section{Project directory layout}
Java projects have a specific directory layout. It is possible to deviate, for instance, when you have
a build tool that enforces a different layout. But we don't recommend this, especially because we
want to have a structure that is easy to understand also for \emph{others}.

\subsection{Global setup and shared directories}
Projects reside inside their own directory. Build tools like Netbeans or Eclipse are able to ``link'' such projects,
 for instance by importing jar files from one project into the other,
 or by automatic rebuilding projects on which your current project depends.
 Our build scheme allows for similar project dependencies, as described above, but does not integrate at the level of source code. As a consequence, you can allocate your project directory wherever you want, but for the sake of sharing
 libraries, sharing data, or sharing build scripts, we need a few \emph{shared} directories.
 ``Shared'' here means mostly sharing among your own projects; it's not a directory within a shared filesystem.
 Real sharing is done by means of shared git repositories. (See below).
 These shared directories are assumed to be located in a common base directory that we will call here
 the ``\verb"JavaProjects>"'' directory.


You are free to chose the location for \verb"<JavaProjects>", for example, it could be
\verb"C:\JavaProjects" on some Windows7 system or \verb"/user/youraccount/javaprogs" on some Linux system.

Within \verb"<JavaProjects>"  we (currently) have a few directories.
The most important one is ``\verb"hmibuild"'', as it contains the shared \verb"ant" scripts. 

You can also choose the location of individual project directories wherever you want,
as long as you specify that location \emph{relative to the } \verb"JavaProjects>" base.
Each project has an \verb"ant" build file called \verb"build.xml", and an\verb"built.properties" file
located in that project's directory.
The \verb"build.xml" file is usually a clone of the following:
\begin{verbatim}
<?xml version="1.0" encoding="UTF-8"?>
<project name="MyApp"  default="run">
   <import file="../hmibuild/build.xml" />
</project>
\end{verbatim}

This example assumes that your project is located next to your copy of \verb"hmibuild'', so that the file path \verb"../hmibuild/build.xml"
is actually ok. 

\subsection{Inside the project directory}

In principle, we have a fixed directory layout within a project:

\begin{verbatim}
<JavaProjects>
   <YourProject>
       build.xml
       build.properties
       ivy.xml
       src
          <Java source files for project packages>
       test
          <Java Source Files for (JUnit) tests>
          lib
             <jar files for testing, like junit-4.8.jar>
       lib
          <library files, i.e. jar files, .dll or .so files etcetera>
       resource
          <data files, like configuration files, icon images, etcetera>
       docs
          <project documentation, including generated Javadoc files>
       build
          <class files generated by compiling this project>
       dist
          <jar files, containing class files and resources for this project>
\end{verbatim}

\subsection{Sharing by means of GIT repositories}
Projects can be shared via one of our GIT repositories.
The Hmi related packages, as well as other shared material are found on the
\verb"hmigit.ewi.utwente.nl/" repositories.
Access to these repositories are available on request; normally, you don't want the source code anyway, and you should use
the packaged jar files, either stable releases, or from the daily build, via the ``\verb"ant resolve"'' method.

What should be present \emph{on a GIT repository}?
\begin{itemize}
\item Source files, data files, documentation: yes.
\item Binaries, and class files: no.
\item Library files, i.e. dll's, jar files etcetera: no.
\item Netbeans or Eclipse project directories or files: no.
\item The \verb"ant" \verb"build.xml" file and  \verb"build.properties" file: yes.
\end{itemize}

\subsection{Project building using ant }

We assume here that you are using the commandline version of \verb"ant". (There are tools
like Netbeans or UltraEdit that allow you to issue these \verb"ant" command from within the tool.)

Tip: \verb"ant" is by default rather ``verbose'', showing for instance all sub-targets as they are called
from a main target. To get rid of this, you can add a system environment variable
called \verb"ANT_ARGS" with value \verb"-quiet" to your system. (In Windows this is done via
the ``Advanced'' System properties panel.)

\begin{itemize}
\item As always, \verb#ant -p# prints all available targets.
\item \verb#ant resolve# tries to check the current contents of the project's \verb#lib#
directory with similar directories within the repository. If necessary, repository files are copied into
the project's \verb#lib# directory, either because they were missing, or because they were
older than the repository version. Which files will be checked must be specified
within the build.properties file; see below.
\item \verb#ant resolveresources# is very much like \verb#ant resolve#, but compares and copies
\emph{resource files}, that is, the contents of the \verb#resource# directory.

\item Simply typing ``\verb#ant#'' should compile if necessary, then run the (current) main class of the project.
This class is specified inside \verb#build.properties# by means of the \verb#run.main.class# variable.
You can change by running \verb"ant main". This shows the current main class, and asks for a new main class.
Simply typing a return will retain the current main).

\item \verb"ant doc" produces JavaDoc documentation inside the \verb"docs/modulename" directory,
where modulename is specified within \verb"build.properties".
The packages to be included in the javadoc arespecifies also in
\verb"build.properties".

\item \verb#ant main# shows current main class, then asks for new main class. (Simply \verb#<Return># won't change it).
(There is a similar target \verb#ueSetmain#, used in combination with tools like UltraEdit)
\item \verb#ant doc# produces JavaDoc documentation within the project's \verb#docs# directory,
including a zipped version inside the \verb#dist# directory.
\item \verb#ant clean# deletes the \verb#build# directory, including all class files.
\item \verb#ant jar# produces a \verb"jar" file containing clas files and resources inside the \verb#dist# directory.


\item  \verb"ant release", and \verb"ant majorrelease"
create  a new minor or major release. 
First a new jar is created, then this jar file is copied into the repository directory where the project
 publishes its releases.
It does \emph{not} upload anything to the repositories.
See the discussion on repositories below.

%\item \verb"ant newversion" increments the ``micro'' part of the ``specification version''
%from the \verb"manifest.mf" file. (The ``micro'' part is the ``third'' number in a version like \verb"2.1.4").


\item \verb"ant rebuild" is a special purpose version of the standard \verb"ant build" target.
It relies on the \verb"rebuild-list" property, specified in \verb"build.properties".
The latter provides a list of project directories, of projects on which the current project depends, directly or indirectly.
\item \verb"ant test" runs (JUnit) test cases for the project as a whole.
\end{itemize}


\subsection{The build.properties file}
The \verb"build.properties" file contains a number of lines of the form \verb"ant-property=value".
Most of these properties are set inside the \verb"ant" files to default values, but when you set
them explicitly  inside \verb"built.properties", then they will overrule the defaults.
The minimal \verb"built.properties" file would contain a line similar to:
\verb"shared.project.root=..",
and a line like: \verb"run.main.class=package.main".

\begin{itemize}
\item \verb"shared.project.root": the relative path to the \verb"<JavaProjects>" directory.
(This one is important; with an incorrect path nothing will work)\\
Example: \verb"shared.project.root=.."
\item \verb"run.main.class": The name of the Java class that should be run as ``main'' class.
Note that you must specify the fully qualified classname, that is, including the package name.
Example: \verb"run.main.class=hmi.util.Info".
\item \verb"resource.path": when set, it specifies a path (as always, relative to \verb"<JavaProjects>"),
to a directory that will be included on the Java classpath. Directories found on this path are suitable
for loading resources, for instance using the \verb"hmi.util.Resources" class, or via
Java methods like \verb"getResourceAsStream".
Example:\\
\verb"resource.path=${shared.project.root}/Shared/repository/Humanoids"
\item \verb"run.jvmargs":  (extra) arguments for \verb"java", while running the main class.
Example: \verb"run.jvmargs=-Xms128m -Xmx1024m"
\item \verb"run.library.path": Usually the java library path contains just the project's \verb"lib" directory,
so for instance \verb"dll" files should reside in that directory.
When you want to include other places, you can specify this with the \verb"run.library.path" property.
Example:\\ \verb"run.library.path=lib;${env.windir}/system32"
\item \verb"javadoc.packages": the packages to be included when running javadoc,
via \verb"ant doc". Example: \verb"javadoc.packages=hmi.graphics.*"
\item \verb"module.name": The (base-)name for \verb"jar" files for this project, and also used for the javadoc directory inside the project's \verb"doc" directory.
The actual version of a \verb"jar" file is determined from the ``Specification-Version'' line inside
the \verb"manifest.mf" file.
\item \verb"dependencies" and \verb"resources": a list of repository modules for the project's \verb"lib" directory
or \verb"resource" directory.
These modules will be updated when running \verb"ant resolve" (for the the ``dependencies''), or
when running \verb"ant resolveresources" (for the ``resources'').
Note that you can sometimes refer to \emph{external} resources via the \verb"resource.path" property, rather than
resolving them into your \verb"resource" directory.
  \begin{itemize}
    \item The line of the form \verb#dependencies=<dep1>, <dep2>, .....,<dep-n># specifies \emph{library}
      dependencies, that is, files that live within the project's \verb#lib# directory.
    \item The line of the form \verb#resources=<dep1>, <dep2>, .....,<dep-n># specifies \emph{resource}
      dependencies, that is, files that live within the project's \verb#resource# directory.
    \item A dependency (i.e. an entry of the dependencies line) refers to some
      directory within the repository. The contents of that directory will be compared and copied if necessary.
    \item A dependency of the form \verb#<organization>/<module.name>#, or of the form
      \verb#<module.name># for projects that don't specify an ``organization'', refers
      to the ``\emph{latest stable}'' release build on the repository.
    \item A dependency of the form \verb#<organization>/<module.name>/<version># or
      \verb#<module.name>/<version># refers to some \emph{specified version} within the repository.
      The allowed ``\verb#<version>#'' values are: \verb#latest#, \verb#alpha#, \verb#beta#, or
      some version \emph{number} like \verb#2.1#, \verb#-3.1.4# or \verb#_2#.
  \end{itemize}
Example:\\
\verb"dependencies=Sun/jogl, Hmi/HmiGraphics/beta, Hmi/HmiMath/1.1"
\item \verb"repositories": A list of directories that act as repository, for the purpose of
\verb"ant resolve" and \verb"ant resolveresources". The default repository list includes \verb"HmiShared/repository"
and \verb"Shared/repository", so usually it isn't necessary to specify this property. Example:\\
\verb"repositories=HmiShared/repository, /C:/Myrepository"

\item \verb"publish.repository": The path to the repository directory where \emph{this} project
will publish its releases. The default place is \verb"Shared/repository".
Example: \verb"publish.repository=/C:/MyRepos"
\item \verb"organization": When specified, organization will be a subdirectory within \verb"publish.directory"
where releases are published. Example: \verb"organization=Hmi"
\item \verb"rebuild-list": a list of project directories (relative to \verb"<JavaProjects>"),
used by the \verb"ant rebuild" target.
\end{itemize}


\noindent
Our \verb"ant" \verb"build.xml" files are in essence identical for all projects.
Occasionally, you might have to change the line:\\
\verb#<property name="shared.project.root" location=".."  />#\\
The line above specifies that our shared root directory ``\verb"<JavaProjects>"'' is one level up
from the project directory, which is correct for most projects.
If your project directory is \emph{not} a direct subdirectory of \verb"<JavaProjects>", you
should adapt this line inside \verb"build.xml" accordingly.
Customization can be dome by editing the \verb"build.properties" file, that we describe below. If this is not enough,
you can create a \verb"build-customize.xml" file inside your project directory,
containing extra \verb"ant" targets as you see fit.
An example \verb"build.properties" file:

\section{Repositories}

A ``repository'' is a collection of library files and data.
We have such repositories in the form of local repository directories, as well as (remote) SVN repositories.
The local directories can be seen as largely as ``cached'' copies of the corresponding SVN versions.
For each project, we maintain a module inside a repository, where various releases are kept, each
in its own subdirectory.
For instance, the \verb"HmiUtil" module contains the \verb"jar" files for the \verb"hmi.util" package.
The ``publish repository'' for the \verb"HmiUtil" project, specified inside its \verb"build.properties" file,
is \verb"HmiShared/repository/HMI".
Therefore, the releases are maintained within subdirectories of
\verb"<JavaProjects>/HmiShared/repository/HMI/HmiUtil".
Inside that directory you will find subdirectories \verb"alpha", \verb"beta", \verb"latest",
and a few more carrying explicit version numbers, like \verb"1.0".
The idea is that ``\verb"latest"'' always contains the latest (stable) release version.
Inside the directory, you will typically find a \verb"lib" directory with content that will be added or
updated for an \verb"ant resolve" action.

\section{Versioning}
Projects can produce releases in the form of packaged \verb"jar" files, carrying some
version number, like for instance \verb"HmiUtil-1.1.jar".
This version number is also present within the jar's \emph{manifest}, which is kept
in the \verb"manifest.mf" file within the project directory.
A version number has the form \verb"<major>.<minor>.<micro>", (where the \verb"<micro>" part is optional).
There are no strict rules for dealing with version numbering, but the following guidelines
are sensible:
\begin{enumerate}
\item When a new version is produced that does \emph{not} introduce or modify existing
functionality, the specification version remains the same, and only the implementation version string
is updated. (We automatically use the current date and time for the implementation version string.)
\item When functionality is \emph{added}, the minor version of the specification version is incremented.
\item For more substantial modifications, in particular removal of functionality, or a complete redesign,
the major version is incremented (and the minor version goes to zero).
\item  There is no clear role for the micro version number. Typically, one can use this
for small changes in functionality, especially when new beta releases are made.
They are also used when a ``stable'' release turns out to contain bugs, in which case
improved versions get a new micro version.
\item The \verb"manifest.mf" file also includes an ``Implementation Version'' line,
which specifies the date and time of the release. Implementation versions should not be \emph{compared},
i.e. a ``later'' implementation should not be considered ``better'', just ``different''.
These implementation version lines are updated automatically when a new jar file is produced.
\end{enumerate}

The micro version number of a project can be incremented by running \verb"ant newversion".
The idea is to increment the micro version of beta releases often, and semi-automatic.
Increments of the major or minor version should be done \emph{manually}, by editing the \verb"manifest.mf" file.
Versions of the manifest inside a \verb"jar" file can be queried at runtime, for instance, to check
that some library \verb"jar" file is not out-of-date. For this purpose, we often add a special \verb"Info" class
to every package. When this is specified as the ``main'' class, it will print the manifest
information, at least when run from a \verb"jar" file.
This \verb"Info" class contains methods for easy checking version compatibility.
For example, \verb#hmi.util.Info.requireVersion("1.1")# will show an error message when
the older \verb"HmiUtil-1.0.jar" version happens to be present in the \verb"lib" directory.
(If you copy the \verb"Info" class, you should change just the \emph{package} line inside the source)

